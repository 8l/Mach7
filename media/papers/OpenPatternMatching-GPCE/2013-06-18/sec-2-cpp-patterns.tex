\section{Pattern Matching in \Cpp{}} %%%%%%%%%%%%%%%%%%%%%%%%%%%%%%%%%%%%%%%%%%
\label{sec:cpppat}


%While our library is not limited to a predefined set of patterns, it provides an 
%implementation of common patterns seen in the functional languages. We introduce
%them here as an example of the syntax available to the user, while we leave the 
%details of their implementation to \textsection\ref{sec:impl}.

The object analyzed through pattern matching is called \term{subject}, while its 
static type -- \subterm{type}{subject type}.
Consider for example the following definition of factorial in \emph{Mach7}:

\begin{lstlisting}[keepspaces]
int factorial(int n) {
  unsigned short m;
  Match(n) {
    Case(0)  return 1;
    Case(m) return m*factorial(m-1);
    Case(_)  throw std::invalid_argument("factorial");
  } EndMatch
}
\end{lstlisting}

\noindent
The subject \code{n} is passed as an argument to \code{Match}-statement and is 
then analyzed through \code{Case}-clauses listing various patterns. In the 
\subterm{case analysis}{first-fit} strategy typically adopted by functional languages, the 
matching proceeds in sequential order while the patterns guarding their 
respective clauses are \subterm{pattern}{rejected}. Eventually, the statement guarded by the 
first \subterm{pattern}{accepted} pattern is executed or the control reaches the end of 
the \code{Match}-statement.

The value 0 in the first case clause is an example of a \subterm{pattern}{value pattern}. It 
will match only when the subject \code{n} is 0. The variable \code{m} in the 
second case clause is an example of a \subterm{pattern}{variable pattern}. It will bind to 
any value that can be represented by its type. The name \code{_} in the last 
case clause refers to the common instance of the \subterm{pattern}{wildcard pattern}. Value, 
variable and wildcard patterns are typically referred to as \subterm{pattern}{primitive 
patterns}. We extend the list of primitive patterns with a \subterm{pattern}{predicate 
pattern} whereby we allow the use of any unary predicate or nullary 
member-predicate as a pattern: e.g. \code{Case(even) ...} (assuming 
\code{bool even(int);}) or \code{Case([](int m) \{ return m^m-1; \}) ...} for $\lambda$-expressions.


%% In functional languages based on type inference, variable 
%% pattern is also an irrefutable pattern since its type can be inferred from the 
%% pattern-matching context, however, in \Cpp{}, it becomes a refutable pattern 
%% since all the variables have to be pre-declared and the type of the variable may 
%% be different from the type expected by the pattern. In the above example, 
%% variable \code{m} will reject any value of type \code{int} that is not a value
%% of type \code{unsigned short}, hence last case clause is not \subterm{case clause}{redundant}.

%% Pattern matching is closely related to \emph{equational reasoning} and the above 
%% factorial can alternatively be defined as following in Haskell 98~\cite{Haskell98Book}:
%% %
%% \begin{lstlisting}[language=Haskell]
%% factorial 0 = 1
%% factorial (n+1) = (n+1) * factorial n
%% \end{lstlisting}
%% %
%% \noindent
%% The \codehaskell{(n+1)} pattern in the left-hand side of the equation defining 
%% the factorial is an example of \subterm{pattern}{n+k pattern} and an equation in itself: 
%% $n+1=v$. According to its informal semantics ``Matching an $n+k$ pattern (where 
%% $n$ is a variable and $k$ is a positive integer literal) against a value $v$ 
%% succeeds if $v \ge k$, resulting in the binding of $n$ to $v-k$, and fails 
%% otherwise''~\cite{haskell98}. n+k patterns were introduced into Haskell to let 
%% users express inductive functions on natural numbers. Besides succinct notation, 
%% this could facilitate automatic proof of termination of such functions by the 
%% compiler. Numerous debates over semantics and usefulness of the n+k patterns 
%% resulted in their removal from the Haskell 2010~\cite{haskell2010}. 
%% Generalization of n+k patterns, called \subterm{pattern}{application patterns} has been 
%% studied by Oosterhof~\cite{OosterhofThesis}. Application patterns treat n+k 
%% patterns as equations, which matching tries to solve or validate.
%% 
%% We are neutral in the debate of utility of n+k patterns, but would like to point 
%% out that in \Cpp{}, where variables have to be explicitly declared and given a 
%% type, one can use a variable's type to avoid at least some semantic issues. For 
%% example, matching $n+1$ against 0 can be rejected when type of $n$ is 
%% \code{unsigned int}, but accepted (binding $n$ to $-1$) when its type is 
%% \code{int}. For the sake of experiment and as a demonstration that n+k patterns 
%% can be brought on first class principles, \emph{Mach7} provides an 
%% implementation that treats them as application patterns, as demonstrated by the 
%% following function for fast computation of Fibonacci numbers:
%% 
%% \begin{lstlisting}[keepspaces]
%% int fib(int n) {
%%   var<int> mm;
%%   Match(n) {
%%     Case(1)         return 1;     
%%     Case(2)         return 1;
%%     Case(2*mm)     return sqr(fib(mm+1)) - sqr(fib(mm-1));
%%     Case(2*mm+1) return sqr(fib(mm+1)) + sqr(fib(mm));
%%   } EndMatch
%% }
%% \end{lstlisting}
%% 
%% \noindent
%% Note that the variable $m$ is now declared with library's type \code{var<int>} 
%% instead of a simple \code{int}. This is necessary because in the context of 
%% pattern matching induced by the \code{Case}-clause we would like to capture the 
%% structure of the expression \code{2*mm+1} instead of just computing its value. 
%% Type \code{var<T>} facilitates exactly that in the pattern-matching context. In 
%% non-pattern-matching contexts, like the return statement above, it provides an 
%% implicit conversion to type \code{T} and thus behaves as a value of type \code{T}. 
%In the rest of this paper, we use cursive font on all the variables of 
%library's type in order to help the reader visually distinguish them from the 
%variables of built-in or any non-\emph{Mach7} types.

%% Since the underlying type of variable $m$ is still \code{int}, the third case 
%% clause will only match even numbers $n$, binding the value of $m$ to 
%% $\frac{n}{2}$, while the fourth clause will only match odd numbers $n$, binding 
%% the value of $m$ to $\frac{n-1}{2}$.  Note that in these clauses the subject to  
%% a variable pattern $m$ is not the original subject $n$ anymore, but 
%% $\frac{n}{2}$ and $\frac{n-1}{2}$ respectively. Matching against even more 
%% complex expressions is discussed in \textsection\ref{sec:slv}, where we do not 
%% restrict ourselves with the equational interpretation, and instead offer an alternative 
%% point of view on n+k patterns that let the user choose a suitable semantics.  

A \emph{guard} is a predicate attached to a pattern that may make use of the 
variables bound in it. The result of its evaluation will determine whether the 
case clause and the body associated with it will be \emph{accepted} or 
\emph{rejected}. Combining patterns with guards gives rise to \subterm{pattern}{guard 
patterns}, which in \emph{Mach7} are expressions of form $P$\code{|=}$E$, where $P$ is a 
pattern and $E$ is an expression guarding it.
%%  As with n+k patterns, the variables 
%% involved in the condition must be of the library's type \code{var<T>} to allow 
%% for lazy evaluation of condition at match time. For example, a case clause of 
%% the form \code{Case(mm |= mm\%2==1)} in  the example above will match any odd value 
%% $n$, binding $m$ to $n$. Similarly, a case clause of the form \code{Case(3*mm |= mm\%2==0)} 
%% will only match values $n$ that are divisible by 6, while binding $m$ to $\frac{n}{3}$.

%% Often times, guard patterns take the form that checks equality or inequality of 
%% two variables bound via pattern matching: e.g. \code{Case(xx, yy |= yy==xx)} is 
%% a two-arument case clause taking a variable and guard patterns to check that 
%% both subjects are the same. 
%% When there are many of them, the code quickly becomes obscure and hard to 
%% understand. Naturally, we would prefer to write something like \code{Case(xx,xx)} 
%% with the semantics of \subterm{pattern}{equivalence patterns}, which match only when all the 
%% independent uses of variable $x$ get bound to the same value. 
%% Neither OCaml nor Haskell support equivalence patterns, while Miranda and Tom's 
%% extension~\cite{Moreau:2003} of C, Java and Eiffel does. Grace and Thorn 
%% approximate the simplicity of equivalence patterns by explicitly marking the 
%% non-binding uses of a variable. We follow the same approach in \emph{Mach7} -- a unary 
%% $+$ placed in front of a variable (or an n+k pattern) turns it into an 
%% expression evaluated at match time that behaves as a value pattern. As an 
%% example, consider the following \emph{Mach7} implementation of the slower 
%% Euclidian algorithm with subtractions:
%% 
%% \begin{lstlisting}[keepspaces,columns=flexible]
%% unsigned int gcd(unsigned int a, unsigned int b) {
%%   var<unsigned int> xx, yy;
%%   Match(a,b) {
%%     Case(xx,+xx)     return xx;
%%     Case(xx,+xx+yy) return gcd(xx,yy);
%%     Case(xx,+xx-yy)  return gcd(xx-yy,yy);
%%   } EndMatch
%% }
%% \end{lstlisting}

\noindent
%% The subjects in \emph{Mach7} are matched left to right, so the value of $x$ is 
%% bound before its value gets used in the equivalence pattern $+x$ in the first 
%% clause. Patterns $+x+y$ and $+x-y$ in the second and third clauses are then 
%% simply n+k patterns with respect to variable $y$ and a constant value bound in 
%% $x$. %We use parentheses in the second clause only for clarity as many novices 
%complained it was hard to decipher the pattern. The meaning is the same as in 
%the other two case clauses since unary $+$ has higher priority than the binary 
%one.

%% Unary $+$ in a pattern expression $+x$ is an example of \term{pattern combinator} 
%% -- an operation that allows composing patterns to produce a new pattern. 
%% Besides \subterm{pattern combinator}{equivalence combinator}, other typical pattern combinators 
%% supported by many languages are \emph{conjunction}, \emph{disjunction} and 
%% \emph{negation} combinators. \subterm{pattern combinator}{Conjunction combinator} in \emph{Mach7} is 
%% predictably represented by \code{P1 && P2} and succeeds only when both patterns 
%% \code{P1} and \code{P2} accept the subject. The set of variables bound by the 
%% resulting pattern is the union of variables bound by each of the patterns. 
%% Similarly, \subterm{pattern combinator}{disjunction combinator} is represented by \code{P1 || P2} and 
%% succeeds when at least one pattern accepts the subject. Languages supporting 
%% disjunction combinator traditionally either require the set of variables bound 
%% by each pattern to be the same or only make the intersection of bound variables 
%% available after the match. In a library solution, we cannot enforce any of these 
%% requirements so it becomes user's responsibility to ensure she only uses 
%% variables from the intersection of bound variables. \subterm{pattern combinator}{Negation combinator} 
%% \code{!P1} binds nothing and succeeds only when pattern \code{P1} is rejected for 
%% a given subject. Negation combinator can be combined with equivalence combinator 
%% to check two subjects for inequivalence: \code{xx} vs. \code{!+xx}. Note that 
%% simply writing \code{!xx} would not have the desired semantics of ``anything but 
%% $x$'', because it will only match values outside the domain of type of 
%% \code{xx}, and if the subject is of the same type as \code{xx}, the pattern 
%% \code{!xx} will never be accepted.
%% 
%% We add two non-standard combinators that reflect the specifics of \Cpp{} -- 
%% presence of pointers and references in the language. \subterm{pattern combinator}{Address combinator} 
%% \code{&P1} can be used to match against the subjects of type \code{T*} when the 
%% pattern \code{P1} expects subjects of type \code{T&}. It fails when the subject 
%% is \code{nullptr} and otherwise forwards the match to pattern \code{P1} applied 
%% to dereferenced subject. The utility of this combinator becomes obvious once we 
%% realize that most of the time we need to match against values pointed to by a 
%% pointer instead of matching against the pointer itself. The alternative would 
%% have been to require patterns match against both pointer and reference types, 
%% however for some cases, it produces non-intuitive ambiguity errors and we decided 
%% to require the user be more explicit instead. Dual to address combinator 
%% is \subterm{pattern combinator}{dereferencing combinator} written as \code{*P1} that can only match 
%% against l-values and forwards the matching to pattern \code{P1} applied to the 
%% address of the l-value. It is less frequently used than the address combinator is, 
%% nevertheless, it is indispensable when we need to obtain an address of pattern's 
%% subject, as that subject may not be easily accessible otherwise.

%Patterns that permit use of the same variable multiple times in them are called 
%\emph{equivalence patterns}, while the requirement of absence of such patterns  
%in a language is called \emph{linearity}. 

Pattern matching is also closely related to \term{algebraic data types} -- a 
possibly recursive \term{sum type} of \term{product types}. In ML and Haskell, an 
\term{Algebraic Data Type} is a data type each of whose values are picked from a 
disjoint sum of data types, called \term{variants}. Each variant is a product 
type, marked with a unique symbolic constant called a \term{constructor}. 
Constructors provide a convenient way of creating a value of its variant type as 
well as discriminating among variants through pattern matching. In particular,
given an algebraic data type $D = C_1(T_{11},...,T_{1m_1}) | \cdots | C_k(T_{k1},...,T_{km_k})$
an expression of the form $C_i(x_1,...,x_{m_i})$ in a non-pattern-matching 
context is called a \subterm{constructor}{value constructor} and refers to a value of type $D$ 
created via constructor $C_i$ and arguments $x_1,...,x_{m_i}$. The same 
expression in the pattern-matching context is called \subterm{pattern}{constructor pattern} 
and is used to check whether the subject is of type $D$ and was created with 
constructor $C_i$. If so, it binds the actual values it was constructed with to  
variables $x_1,...,x_{m_i}$ (or matches against nested patterns if $x_j$ are 
other kinds of patterns).

\Cpp{} does not provide a direct support for algebraic data types, however they 
can be encoded in the language in a number of ways. Common 
object-oriented encodings employ an abstract class to represent the 
algebraic data type and derived classes to represent variants. Consider for 
example the following representation of terms in $\lambda$-calculus in \Cpp{}:

\begin{lstlisting}[columns=flexible]
struct Term       { virtual @$\sim$@Term() {} };
struct Var : Term { std::string name; };
struct Abs : Term { Var&  var;  Term& body; };
struct App : Term { Term& func; Term& arg; };
\end{lstlisting}

\noindent
\Cpp{} allows a class to have several constructors and does not allow 
overloading the meaning of construction for the use in pattern matching. This is
why in \emph{Mach7} we have to be slightly more explicit about constructor patterns, 
which take form \code{C<Ti>(}$P_1,...,P_{m_i}$\code{)}, where $T_i$ is the name of 
the user-defined type we are decomposing and $P_1,...,P_{m_i}$ are patterns that 
will be matched against members of $T_i$ (\textsection\ref{sec:bnd}). ``C'' was
chosen to abbreviate ``Constructor pattern'' or ``Case class'' as its use 
resembles the use of case classes in Scala~\cite{Scala2nd}.
For example, we can write a complete 
recursive implementation of an equality of two lambda terms:

\begin{lstlisting}[columns=flexible]
bool operator==(const Term& left, const Term& right) {
  var<const std::string&> Ss; var<const Term&> xx,yy;
  Match(left          , right          ) {
    Case(C<Var>(Ss)   , C<Var>(+Ss)    ) return true;
    Case(C<Abs>(xx,yy), C<Abs>(+xx,+yy)) return true;
    Case(C<App>(xx,yy), C<App>(+xx,+yy)) return true;
    Otherwise()                          return false;
  } EndMatch
}
\end{lstlisting}

\noindent
This \code{==} is an example of a \term{binary method} -- an operation that 
requires both arguments to have the same type~\cite{BCCLP95}. In each of the 
case clauses we check that both subjects are of the same dynamic type through 
the use of constructor pattern. We then decompose both subjects into 
components and compare them for equality with a combination of a variable pattern 
and an equivalence combinator $+$ applied to the same variable pattern. The use 
of equivalence combinator turn binding use of a variable pattern into a 
non-binding use of that variable's current value as a value pattern.

In general, \term{pattern combinator} is an operation on patterns to produce a 
new pattern. Other typical pattern combinators supported by many languages are
\emph{conjunction}, \emph{disjunction} and \emph{negation} combinators with 
intuitive boolean interpretation. We add few non-standard combinators to 
\emph{Mach7} that reflect the specifics of \Cpp{}, e.g. presence of pointers and 
references in the language.

%The first case 
%clause can be interpreted as following: if both subjects are instances of class 
%\code{Var}, bind the name of the first one to variable $s$ and then check, that 
%the name of the other variable is the same through equivalence combinator: 
%\code{+Ss}. To understand the second and third clauses, note that the 
%corresponding members of classes \code{Abs} and \code{App} are pointers to 
%terms, while to ensure equality we are interested in checking deep equality of 
%the terms themselves. Thus instead of binding pointers to variables, checking 
%they are not \code{nullptr} and forwarding the call to the values they point to 
%etc., we declare our pattern-matching variables of type \code{var<const Term&>} 
%and apply an address combinator to them to be able to match against pointer 
%type. Passing variables without address combinator will not type check because 
%the value bound in each position is of pointer type, while the variable pattern 
%expects the reference type. Note also that unlike the variable patterns of type 
%\code{var<T>} that contain the bound value of type \code{T}, variable patterns 
%of type \code{var<T&>} do not contain the value and only bind the reference to a 
%memory location holding an actual value during matching. This is important for 
%polymorphic classes like \code{Term} to avoid slicing. The right argument is 
%matched in a similar manner: the constructor pattern ensures the dynamic type is 
%\code{Abs} or \code{App} respectively, after which the address combinator 
%ensures the actual pointers matched are not \code{nullptr}, forwarding match to 
%equivalence pattern applied to a variable of underlying type \code{const Term&}, 
%which in turn recursively calls this equality operator on sub-terms.

%There are two critical differences between algebraic data types and classes in 
%object-oriented languages: definition of an algebraic data type is \emph{closed}, 
%while its variants are \emph{disjoint}. Closedness means that once we have listed 
%all the variants, a given algebraic data type may have, we cannot extend it with 
%new variants without modifying its definition. Disjointedness means that a value 
%of an algebraic data type belongs to exactly one of its variants. Neither is the 
%case in object-oriented languages. Classes are \emph{extensible}, since new 
%variants can be added through subclassing, as well as \emph{hierarchical}, since 
%variants are not necessarily disjoint and can form subtyping relation between 
%themselves.
%
%Closedness of algebraic data types is particularly useful for reasoning about 
%programs by case analysis and allows the compiler to perform \emph{exhaustiveness 
%checking} -- a test of whether a given match statement covers all possible 
%cases. Unfortunately, similar reasoning about programs involving extensible data 
%types is necessarily more involved as we are dealing with potentially open set of variants. 
%Exhaustiveness checking in such scenario reduces to checking presence of a case 
%that handles the static type of the subject. Absence of such a case, however, 
%does not necessarily imply inexhaustiveness, only potential inexhaustiveness, 
%as the answer will depend on the actual set of variants available at run-time. 
%
%A related notion of \emph{redundancy checking} arises from the tradition of 
%using \emph{first-fit} semantics in pattern matching. It warns the user of any 
%case clause inside a match statement that will never be entered because of a 
%preceding one being more general. Hierarchical nature of classes has to be 
%further taken into account for redundancy checking as (constructor) patterns 
%accepting a base class should also accept any of its derived classes.

The equality operator on $\lambda$-terms demonstrates nesting of patterns.
The variable pattern was nested within 
an equivalence pattern, which in turn was nested inside a constructor pattern. 
Nesting allows one to combine many different patterns in many different ways, 
which makes them particularly expressive. Here is a well-known functional 
solution to balancing red-black trees with pattern matching due to Chris 
Okasaki~\cite[\textsection 3.3]{okasaki1999purely} implemented in \emph{Mach7}:

\begin{lstlisting}
class T { enum color {black,red} col; T* left; K key; T* right; };
@\halfline@
T* balance(T::color clr, T* left, const K& key, T* right) {
  const T::color B = T::black, R = T::red;
  var<T*> aa, bb, Cc, Dd; var<K&> xx, yy, zz; T::color col;
@\halfline@
  Match(clr, left, key, right) {
    Case(B, C<T>(R, C<T>(R, aa, xx, bb), yy, Cc), zz, Dd) ...
    Case(B, C<T>(R, aa, xx, C<T>(R, bb, yy, Cc)), zz, Dd) ...
    Case(B, aa, xx, C<T>(R, C<T>(R, bb, yy, Cc), zz, Dd)) ...
    Case(B, aa, xx, C<T>(R, bb, yy, C<T>(R, Cc, zz, Dd))) ...
    Case(col, aa, xx, bb) return new T{col, aa, xx, bb};
  } EndMatch
}
\end{lstlisting}

\noindent
The \code{...} in the first four case clauses above stands for \\ %the statement 
\code{return new T\{R, new T\{B,aa,xx,bb\}, yy, new T\{B,Cc,zz,Dd\}\};}. %Class 
%\code{T} implements a tree-node type. For the details of how exactly the 
%balancing by pattern matching works, we refer the reader to Okasaki's original 
%manuscript~\cite[\textsection 3.3]{okasaki1999purely}.

To demonstrate the openness of the library, we implemented numerous 
specialized patterns that often appear in practice and are even built into some 
languages. For example, the following combination of regular-expression and one-of 
patterns can be used to match against a string and see whether the string 
represents a toll-free phone number. 

\begin{lstlisting}
rex("([0-9]+)-([0-9]+)-([0-9]+)", any({800,888,877}), n, m)
\end{lstlisting}

\noindent
The regular-expression pattern takes a \Cpp{}11 regular expression and an 
arbitrary number of sub-patterns. It then uses matching groups to match against 
the sub-patterns. A one-of pattern simply takes an initializer-list with a set of 
values and checks that the subject matches one of them. The variables 
\code{n} and \code{m} are integers, and the values of the last two parts of the pattern will be assigned to them. The parsing is generic 
and will work with any data type that can be read from an input stream -- a 
common idiom in \Cpp{}. Should we also need the exact area code, we can mix in a 
variable pattern with conjunction combinator: \code{a && any(...)}.

%In functional programming most of the data structures are recursive
