\section{Related Work} %%%%%%%%%%%%%%%%%%%%%%%%%%%%%%%%%%%%%%%%%%%%%%%%%%%%%%%%%
\label{sec:rw}

\emph{Extensible Visitors with Default Cases}~\cite[\textsection 
4.2]{Zenger:2001} attempt to solve the extensibility problem of visitors; 
however, the solution has problems of its own. The visitation interface 
hierarchy can easily be grown linearly, but independent extensions by different  
authorities require developer's intervention. On top of the double dispatch the 
solution will incur two additional virtual calls and a dynamic cast for each 
level of visitor extension. The solution is simpler with virtual inheritance, 
which adds even more indirections.

L\"{o}h and Hinze proposed to extend Haskell's type system with open data types 
and open functions~\cite{LohHinze2006}. The solution allows top-level data types 
and functions to be marked as open with concrete variants and overloads defined 
anywhere in the program. The semantics of open extension is given by 
transformation into a single module, which assumes a whole-program view and thus 
is not an open solution unfortunately. Besides, open data types are extensible but not 
hierarchical, which avoids the problems discussed here.

Polymorphic variants in OCaml~\cite{garrigue-98} allow the addition of new 
variants as well as define subtyping on them. The subtyping, however, is not 
defined between the variants, but between combinations of them. 
This maintains disjointness between values from different variants and makes an 
important distinction between \emph{extensible sum types} like polymorphic 
variants and \emph{extensible hierarchical sum types} like classes. Our 
memoization device can be used to implement pattern matching on polymorphic 
variants.

\emph{Tom} is a pattern-matching compiler that can be used together with Java, C or 
Eiffel to bring a common pattern matching and term rewriting syntax into the 
languages~\cite{Moreau:2003}. In comparison to our approach, Tom has much bigger 
goals: the combination of pattern matching, term rewriting and strategies turns 
Tom into a fully fledged tree-transformation language. Its type patterns and \%match 
statement can be used as a type switch; however, Tom's handling of type 
switching is based on decision trees and an \code{instanceof}-like predicate, 
which are inefficient.

Pattern matching in Scala~\cite{Scala2nd} also supports type switching through 
type patterns. The language supports extensible and hierarchical data types, but
their handling in a type switching constructs varies. Sealed classes are handled 
with an efficient switch over all tags, while extensible classes are similarly 
approached with a combination of an \code{InstanceOf} operator and a decision 
tree~\cite{EmirThesis}.