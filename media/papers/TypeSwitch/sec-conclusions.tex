\section{Conclusions} %%%%%%%%%%%%%%%%%%%%%%%%%%%%%%%%%%%%%%%%%%%%%%%%%%%%%%%%%%
\label{sec:cc}

We described three techniques that can be used to implement type switching, type 
testing, pattern matching, predicate dispatching and other facilities that 
depend on a run-time type of an argument.

\emph{Memoization Device} is an optimization technique that maps run-time values 
to execution paths, allowing to take shortcuts on subsequent runs with the same 
value. The technique does not require code duplication and in typical cases adds 
only a single indirect assignment to each of the execution paths. It can be 
combined with other compiler optimizations and is particularly suitable for use 
in a library setting.

\emph{V-Table Pointer Memoization} is technique based on memoization device that 
employs certain properties of virtual table pointers to not only speed up 
execution, but also properly uncover dynamic type of an object. This technique 
is a backbone of our fast type switch as well as memoized dynamic cast 
optimization.

\emph{Tag Precedence List} is yet another technique that can be used to 
implement best-fit type switching on tagged classes. The technique has its pros 
and cons in comparison to v-table memoization, which we discuss in the paper.

These techniques can be used in a compiler and library setting and support well 
separate compilation and dynamic linking. They are open to class extensions and 
interact well with other C++ facilities such as multiple inheritance (including 
repetitive and virtual inheritance) and templates. The techniques are not 
specific to C++ and can be adopted in other languages for similar purposes.

Using the above techniques we implemented a library for efficient type switching 
in C++. Our implementation comes close or outperforms its closest contender -- 
visitor design pattern as well as overcomes the restrictions, inconveniences and 
difficulties in teaching and using, typically associated with it.

We used the library to rewrite an existing code that was relying heavily on 
visitors and discovered that resulting code became much shorter, simpler and easier 
to maintain and comprehend.
