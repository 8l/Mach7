\section{Conclusions and Future Work} %%%%%%%%%%%%%%%%%%%%%%%%%%%%%%%%%%%%%%%%%%%%%%%%%%%%%%%%%%
\label{sec:cc}

Type switching is an open alternative to visitor design pattern that overcomes 
the restrictions, inconveniences, and difficulties in teaching and using, 
typically associated with it. Our implementation of it comes close or 
outperforms the visitor design pattern, which is true even in a library setting 
using a production-quality compiler, where the performance base-line is 
already very high.

%We describe three techniques that can be used to implement type switching, type 
%testing, pattern matching, predicate dispatching, and other facilities that 
%depend on the run-time type of an argument as well as demonstrate their efficiency.
%
%The \emph{Memoization Device} is an optimization technique that maps run-time values 
%to execution paths, allowing to take shortcuts on subsequent runs with the same 
%value. The technique does not require code duplication and in typical cases adds 
%only a single indirect assignment to each of the execution paths. It can be 
%combined with other compiler optimizations and is particularly suitable for use 
%in a library setting.
%
%The \emph{Vtable Pointer Memoization} is a technique based on memoization device that 
%employs uniqueness of virtual table pointers to not only speed up execution, but 
%also properly uncover the dynamic type of an object. This technique is a 
%backbone of our fast type switch as well as memoized dynamic cast optimization.
%
%The \emph{TPL Dispatcher} is yet another technique that can be used to 
%implement best-fit type switching on tagged classes. The technique has its pros 
%and cons in comparison to vtable pointer memoization, which we discuss in the paper.
%
%These techniques can be used in a compiler and library setting, and support well 
%separate compilation and dynamic linking. They are open to class extensions and 
%interact well with other C++ facilities such as multiple inheritance and 
%templates. The techniques are not specific to C++ and can be adopted in other 
%languages for similar purposes.
%
%Using these techniques, we implemented a library for efficient type switching 
%in C++. We used it to rewrite a code that relied heavily on 
%visitors, and discovered that the resulting code became much shorter, simpler, 
%and easier to maintain and comprehend.
