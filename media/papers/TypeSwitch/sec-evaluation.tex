\section{Evaluation} %%%%%%%%%%%%%%%%%%%%%%%%%%%%%%%%%%%%%%%%%%%%%%%%%%%%%%%%%%%
\label{sec:eval}

\begin{figure*}
\begin{tabular}{@{}c@{ }l||@{ }r@{}@{ }r@{}|@{ }r@{}@{ }r@{}||@{ }r@{}@{ }r@{}|@{ }r@{}@{ }r@{}||@{ }r@{}@{ }r@{}|@{ }r@{}@{ }r@{}}
\hline % -----------------------------------------------------------------------------------------------------------------------------------------
\hline % -----------------------------------------------------------------------------------------------------------------------------------------
 &            & \multicolumn{4}{c||}{G++/32}  & \multicolumn{4}{c||}{MS Visual C++/32} & \multicolumn{4}{c}{MS Visual C++/64} \\
\hline % -----------------------------------------------------------------------------------------------------------------------------------------
 & Syntax     & \multicolumn{2}{c|}{Unified} & \multicolumn{2}{c||}{Special} & \multicolumn{2}{c|}{Unified} & \multicolumn{2}{c||}{Special} & \multicolumn{2}{c|}{Unified} & \multicolumn{2}{c}{Special} \\
\hline % -----------------------------------------------------------------------------------------------------------------------------------------
 & Encoding   & \Opn  & \Cls  & \Opn  & \Cls  & \Opn  & \Cls  & \Opn  & \Cls  & \Opn  & \Cls  & \Opn  & \Cls   \\
\hline % -------------------------------------------------------------------------------------------------
\hline % -------------------------------------------------------------------------------------------------
 & Repetitive &\gwNGPp&\gwNGKp&\gwNSPp&\gwNSKp&\VwNGPp&\VwNGKp&\VwNSPp&\VwNSKp&\VxNGPp&\VxNGKp&\VxNSPp&\VxNSKp \\
 & Sequential &\gwNGPq&\gwNGKq&\gwNSPq&\gwNSKq&\VwNGPq&\VwNGKq&\VwNSPq&\VwNSKq&\VxNGPq&\VxNGKq&\VxNSPq&\VxNSKq \\
 & Random     &\gwNGPn&\gwNGKn&\gwNSPn&\gwNSKn&\VwNGPn&\VwNGKn&\VwNSPn&\VwNSKn&\VxNGPn&\VxNGKn&\VxNSPn&\VxNSKn \\
\hline % --------------------------------------------------------------------------------------------------
\multirow{3}{*}{\begin{sideways}{\tiny Forward}\end{sideways}}
 & Repetitive &\gwYGPp&\gwYGKp&\gwYSPp&\gwYSKp&\VwYGPp&\VwYGKp&\VwYSPp&\VwYSKp&\VxYGPp&\VxYGKp&\VxYSPp&\VxYSKp \\
 & Sequential &\gwYGPq&\gwYGKq&\gwYSPq&\gwYSKq&\VwYGPq&\VwYGKq&\VwYSPq&\VwYSKq&\VxYGPq&\VxYGKq&\VxYSPq&\VxYSKq \\
 & Random     &\gwYGPn&\gwYGKn&\gwYSPn&\gwYSKn&\VwYGPn&\VwYGKn&\VwYSPn&\VwYSKn&\VxYGPn&\VxYGKn&\VxYSPn&\VxYSKn \\
\hline % --------------------------------------------------------------------------------------------------
\end{tabular}
\caption{Relative performance of our pattern matching versus visitors. Numbers 
like \f{42} in bold font indicate that our pattern matching is faster than 
visitors by corresponding percentage. Numbers like \s{42} in italics font 
indicate that our solution is slower than visitors (i.e. visitors is faster than 
our solution) by corresponding percentage.}
\label{relperf}
\end{figure*}

In this section, we evaluate the performance of our solution in comparison to its 
de-facto contender -- the visitor design pattern. We also compare performance of 
some typical use cases expressed with our solution and OCaml.

Our evaluation methodology consists of several benchmarks that we believe 
represent various possible uses of objects analyzed with either visitors or 
pattern matching.

The \emph{repetitive} benchmark performs multiple calls on different objects of the 
same most derived type. This scenario happens in object-oriented setting when a 
group of polymorphic objects is created and passed around (e.g. numerous 
particles of a given kind in a particle simulation system). We include it 
because double dispatch becomes about twice faster (27 vs. 53 cycles) in this 
scenario compared to others due to cache and call target prediction mechanisms. 

The \emph{sequential} benchmark effectively uses an object of each derived type only 
once and then moves on to an object of a different type. The cache is typically 
reused the least in this scenario. The scenario is typical of lookup tables, 
where each entry is implemented with a different derived class.

The \emph{random} benchmark is the most representative as it randomly makes calls on 
random objects, which will probably be the most common usage scenario in the 
real world.

The \emph{forwarding} benchmark is not a benchmark on its own, but rather a 
combinator that can be applied to any of the above scenarios. It refers to the 
common technique used by visitors where, for class hierarchies with multiple 
levels of inheritance, the \code{visit} method of a derived class will provide a 
default implementation of forwarding to its immediate base class, which, in turn, 
may forward it to its base class, etc. The use of forwarding in visitors is a 
way to achieve substitutability, which in type switches corresponds to the use 
of base classes in the case clauses.

These benchmarks were executed in the following configurations:

\begin{itemize}
\setlength{\itemsep}{0pt}
\setlength{\parskip}{0pt}
\item Sony VAIO\textsuperscript{\textregistered} laptop with Intel\textsuperscript{\textregistered} Core\texttrademark i5 460M 
      Processor at 2.53 GHz equipped with 6GB of RAM running Windows 7 
      Professional
      \begin{itemize}
      \setlength{\itemsep}{0pt}
      \setlength{\parskip}{0pt}
      \item G++ 4.5.2 under MinGW executed with -O2 and producing x86 binaries
      \item MS Visual C++ 2010 Professional x86/x64 binaries with Profile-Guided Optimizations
      \end{itemize}
\end{itemize}

Each benchmark under each configuration was tested with either \emph{unified} 
or \emph{specialized} syntax. Each syntax was used to run tests on \emph{Open} 
(\emph{polymorphic base class} encoding) and \emph{Tag} (\emph{tag class} 
encoding). Specialized syntax avoids generating unnecessary syntactic structure 
used to unify syntax, and thus produces faster code. We include it in our results as 
a compiler implementation of pattern matching will be able to distinguish each 
case and thus generate only the required structure.

The code on the critical path of our type switch implementation benefits 
significantly from branch hinting as some branches are much more likely than 
others. We use the branch hinting facilities of GCC to tell the compiler which 
branches are more likely, but, unfortunately, Visual C++ does not have similar 
facilities. The official way suggested by Microsoft to achieve the same effect 
is to use \emph{Profile-Guided Optimization} and let the compiler gather 
statistics on each branch. This is why the result for Visual C++ reported here 
are those obtained with profile-guided optimizations enabled. The slightly 
less-favorable-for-us results without profile-guided optimizations can be found 
in the accompanying technical report~\cite{TR}.
%The results of optimizing code created with Visual C++ by using profile 
%guided optimizations as currently Visual C++ does not have means for branch 
%hinting, which are supported by G++ and proven to be very effective in few 
%cruicial places. Profile guided optimization in Visual C++ lets compiler find 
%out experimentally what we would have otherwise hinted, even though this 
%includes other optimizations as well.

We compare the performance of our solution relative to the performance of visitors in 
Figure~\ref{relperf}. The values are given as percentages of performance increase 
against the slower technique. Numbers in bold represent cases where our type 
switching was faster than visitors were. Numbers in italics indicate cases where 
visitors were faster.

From the numbers, we can see that type switching wins by a good margin in 
the presence of at least one level of forwarding on visitors. Using type 
switching on closed hierarchies is also a definite winner.

From the table it may seem that Visual C++ is generating not as good code as GCC 
does, but remember that these numbers are relative, and thus the ratio depends on  
both the performance of virtual calls and the performance of switch statements. Visual 
C++ was generating faster virtual function calls, while GCC was generating 
faster switch statements, which is why their relative performance seem to be much 
more favorable for us in the case of GCC.

Similarly the code for x64 is only slower relatively: the actual time spent for 
both visitors and type switching was smaller than that for x86, but it was much 
smaller for visitors than type switching, which resulted in worse relative 
performance.

\subsection{V-Table Pointer Memoization vs. Tag Precedence List}
\label{sec:cmp}

With a few exceptions for x64 tests, it can be seen from Figure~\ref{relperf} 
that the performance of the tag precedence list approach (the Tag column) dominates the 
performance of the v-table pointer memoization approach (the Open column). We believe 
that the difference, often significant, is the price one pays for the true 
openness of the v-table pointer memoization solution.

Unfortunately, the tag precedence list approach is not truly open. The use of tags, 
even if they would be allocated by compiler, may require integration efforts to 
ensure that different DLLs have not reused the same tags. Randomization of tags,
similar to a proposal of Garrigue~\cite{garrigue-98}, will not eliminate the 
problem and will surely replace jump tables in switches with decision trees. This 
will likely significantly degrade the numbers for the Tag column of 
Figure~\ref{relperf}, since the tags in our experiments were all sequential.

Besides, the tag precedence list approach relies on static cast to obtain the 
proper reference once the most specific case clause has been found. As we 
described in \textsection\ref{sec:vtblmem}, this has severe limitations in the 
presence of multiple inheritance, and thus is not as versatile as the other 
solution. Overcoming this problem will either require the use of 
\code{dynamic_cast} or techniques similar to those we used in v-table pointer 
memoization. This will likely degrade performance numbers for the Tag column even further.

Note also that the v-table pointer memoization approach can be used to implement both
first-fit and best-fit semantics, while the tag precedence list is only suitable 
for best-fit semantics. Their complexity guarantees also differ: v-table pointer 
memoization is constant on average, and slow on the first call. Tag list approach is 
logarithmic in the size of the class hierarchy on average (assuming a balanced 
hierarchy), including on the first call.
