\section{Introduction} %%%%%%%%%%%%%%%%%%%%%%%%%%%%%%%%%%%%%%%%%%%%%%%%%%%%%%%%%
\label{sec:intro}

%Motivate the problem
%Give a summary of the paper: what you did and how
%Explicitly state your contribution

Algebraic data types as seen in functional languages are closed and their 
variants are disjoint. This allows for easy addition of new functions through 
case analysis as well as their efficient implementation. Data types in 
object-oriented languages are extensible and hierarchical (non-disjoint), which 
significantly complicates the addition of new functions. Existing approaches to 
case analysis on such types are either efficient or open, but not both. Truly 
open approaches, which allow for independent extensibility, modular 
type-checking and dynamic linking, rely on class-membership testing combined 
with a decision tree. Efficient approaches rely on sealing either the class 
hierarchy or the set of functions, which looses extensibility. Consider a simple 
expression language:

\begin{lstlisting}
@$exp$ \is{} $val$ \Alt{} $exp+exp$ \Alt{} $exp-exp$ \Alt{} $exp*exp$ \Alt{} $exp/exp$@
\end{lstlisting}

\noindent 
In an object-oriented language without direct support for algebraic data types, 
the type representing an expression-tree in the language will typically be 
encoded as an abstract base class, listing the (sealed set of) allowed virtual 
functions, with derived classes representing variants:

\begin{lstlisting}[keepspaces,columns=flexible]
struct Expr { virtual @$\sim$@Expr() {} };
struct Value  : Expr { int value; };
struct Plus   : Expr { Expr* e1; Expr* e2; }; // ...
\end{lstlisting}

\noindent
A simple evaluator for this language can be implemented with the aid of a
virtual function \code{eval()} declared in the base class \code{Expr}. 
The approach is \emph{intrusive}, as we will have to modify the base class every 
time we would like to add a function. A less-intrusive solution to the 
extensibility of functions can be achieved with \emph{visitor design 
pattern}~\cite{DesignPatterns1993}. That solution, however, is not open as it 
restricts extensibility of classes.

Our solution overcomes extensibility problems of both functions and 
types by allowing external introspection of objects with a type switch:

\begin{lstlisting}[keepspaces,columns=flexible]
int eval(const Expr* e)
{
  Match(e)
    Case(Value,  n)    return n;
    Case(Plus,   a, b) return eval(a) + eval(b);
    Case(Minus,  a, b) return eval(a) - eval(b);
    Case(Times,  a, b) return eval(a) * eval(b);
    Case(Divide, a, b) return eval(a) / eval(b);
  EndMatch
}
\end{lstlisting}

\noindent
The syntax is provided without any external tool support. Instead we rely on a 
few C++0x features~\cite{C++0x}, template meta-programming, and macros. It runs 
about as fast as the OCaml version (\textsection\ref{sec:ocaml}), and, depending 
on the usage scenario, compiler and underlying hardware, comes close or 
outperforms the handcrafted C++ code based on the \emph{visitor design pattern} 
(\textsection\ref{sec:eval}). The mapping of members to matching positions that 
the user has to provide to make the above example fully functional is 
omitted~\cite[\textsection 3.1]{TR}.

%\subsection{Summary}

We present techniques based on memoization (\textsection\ref{sec:copc}) and 
class precedence list (\textsection\ref{sec:cotc}) that can be used to
implement type switching efficiently based on the run-time type of the
argument:
  \begin{itemize}
  \setlength{\itemsep}{0pt}
  \setlength{\parskip}{0pt}
  \item The techniques come close and often outperform its de facto contender -- 
        visitor design pattern -- without sacrificing extensibility (\textsection\ref{sec:eval}).
  \item They work in the presence of multiple inheritance, both repeated and 
  shared, as well as in generic code (\textsection\ref{sec:vtblmem}).
  \item The solution is open by construction (\textsection\ref{sec:poets}), 
        non-intrusive, and avoids the control inversion typical for visitors.
  \item It applies to polymorphic (\textsection\ref{sec:vtp}-\ref{sec:vtblmem}) and 
        tagged (\textsection\ref{sec:cotc}) class hierarchies through a unified  
        syntax~\cite[\textsection 8]{TR}.
  \item Our memoization device (\textsection\ref{sec:memdev}) generalizes to 
        other languages and can be used to implement type switching 
        (\textsection\ref{sec:vtblmem}), type testing 
        (\textsection\ref{sec:poets},\cite[\textsection 9]{TR}), predicate dispatch 
        (\textsection\ref{sec:memdev}), and multiple dispatch efficiently.
  \item We list conditions under which virtual table pointers, commonly used in 
        C++ implementations, uniquely identify the exact subobject within the 
        most derived type (\textsection\ref{sec:vtp}).
  \item We also build an efficient cache indexing function for virtual table 
        pointers that minimizes the amount of conflicts 
        (\textsection\ref{sec:sovtp},\ref{sec:moc},\cite[\textsection 5.3.2]{TR}).
  \end{itemize}

\noindent
A practical benefit of our solution is that it can be used right away with any 
compiler with a descent support of C++0x without requiring the installation of 
any additional tools or preprocessors. The solution is a proof of concept that 
sets a minimum threshold for the performance, brevity, clarity and usefulness of 
a language solution for open type switching in C++.

Due to page restrictions, we will only deal here with the part of the library 
concerned with efficient type testing, type identification, and type switching. 
These primitive operations are essential in providing support for type patterns 
and constructor patterns, and their efficiency relative to the visitor design 
pattern might become a major argument for their adoption. We refer the reader to 
technical report~\cite{TR} for more details on our pattern-matching 
library.

%Section~\ref{sec:probl} will motivate the problem as well as discuss some 
%existing solutions and their limitations. Section~\ref{sec:copc} will outline 
%the three techniques that comprise our solution, while the 
%section~\ref{sec:eval} will evaluate their performance. Related work is 
%discussed in Section~\ref{sec:rw}, followed by concluding remarks in 
%Section~\ref{sec:cc}.
