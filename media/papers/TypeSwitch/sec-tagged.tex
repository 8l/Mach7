\section{Solution for Tagged Classes}
\label{sec:cotc}

Memoization device outlined in \textsection\ref{sec:memdev} can in principle be 
applied to tagged classes too. The dynamic cast will be replaced by a small 
compile-time template meta-program that checks whether class associated with 
given tag is derived from the target type of the case clause. If so, the static 
cast can be used to obtain the offset.

Despite its straightforwardness we felt that it should be possible to do better 
than the general solution, given that each class is already identified with a 
dedicated constant known at compile time.

As we mentioned in \textsection\ref{sec:poets}, nominal subtyping of C++ 
effectively gives every class multiple types. The idea is thus to associate with 
the type not only its most derived tag, but also the tags of all its base classes.
In a compiler implementation such a list can be stored inside the virtual table 
of a class, while in our library solution it is shared between all the instances 
with the same most derived tag in a less efficient global map, associating the 
tag to its tag list.

The list of tags is topologically sorted accordingly to the subtyping relation 
and terminates with a dedicated value distinct from all the tags. We call such a 
list a \emph{Tag Precedence List} (TPL) as it resembles the \emph{Class 
Precedence List} (CPL) of object-oriented descendants of Lisp (e.g. Dylan, 
Flavors, LOOPS, and CLOS) used there for \emph{linearization} of class 
hierarchies. The classes in CPL are ordered from most specific to least specific 
with siblings listed in the \emph{local precedence order} -- the order of the 
direct base classes used in the class definition. TPL is just an implementation 
detail and the only reason we distinguish TPL from CPL is that in C++ classes 
are often separated into interface and implementation classes and it might so 
happen that the same tag is associated by the user with an interface and several 
implementation classes. 

A type switch below, built on top of a hierarchy of tagged classes, proceeds as 
regular switch on the subject's tag. If the jump succeeds, we found an exact 
match, otherwise we get into a default clause that obtains the next tag in the 
tag precedence list and jump back to the beginning of the switch statement for a 
rematch:

\begin{lstlisting}
    const size_t* taglist = 0;
          size_t  attempt = 0;
          size_t  tag     = object->tag;
ReMatch:
    switch (tag) 
    {
    default:
        if (!taglist) 
            taglist = get_taglist(object->tag);
        tag = taglist[++attempt];
        goto ReMatch;
    case end_of_list: break;
    case bindings<D1>::kind_value: s1; break;
    ...
    case bindings<Dn>::kind_value: sn; break;
    }
\end{lstlisting}

\noindent
The above structure lets us dispatch to case clauses of the most derived class 
with an overhead of initializing two local variables compared to a switch on a 
sealed hierarchy. Dispatching to a case clause of a base class will take a time 
roughly proportional to the distance between the matched base class and the most 
derived class in the inheritance graph. When none of the base class tags was 
matched we will necessarily reach the end\_of\_list marker in the tag precedence 
list and thus exit the loop.

Our library automatically builds the function \code{get_taglist} based on 
additional definitions that user provides inside \code{bindings}. The details 
can be found in our accompanying paper\cite{AP}.
