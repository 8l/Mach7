\section{Pattern Matching SELL} %%%%%%%%%%%%%%%%%%%%%%%%%%%%%%%%%%%%%%%%%%%%%%%%
\label{sec:pm}

A Semantically Enhanced Library Language is not a language of its own, but 
rather a sub-language embedded into another language -- C++ in our case. The 
sub-language still has the facilities we typically associate with programming 
languages e.g. syntax, semantics, type system, but they are constrained by
the host language. A particular sub-language can often be implemented in 
different host languages, which is why it is important to describe it 
independently of its host. We thus shall abstract from describing the 
exact syntax and semantics of host-language features that are well understood and documented 
elsewhere~\cite{C++11}.

\subsection{Syntax}
\label{sec:syn}

\begin{figure}[h]
\centering
\begin{tabular}{rcll}
\Rule{Match Statement}     & $M$       & \is{}  & \code{Match(}$e$\code{)} $\{ \left[C s^*\right]^* \}$ \code{EndMatch} \\
\Rule{Case Clause}         & $C$       & \is{}  & \code{Qua(}$T\left[,\varpi\right]^*$\code{)}\Alt{}\code{When(}$\varpi^*$\code{)}\Alt{}\code{Case(}$T\left[,x\right]^*$\code{)}\Alt{}\code{Otherwise(}$x^*$\code{)} \\
\Rule{Target Expression}   & $T$       & \is{}  & $\tau$ \Alt{} $l$ \\
\Rule{Layout}              & $l$       & \is{}  & $c^{\mathsf{int}}$ \\
\Rule{Match Expression}    & $m$       & \is{}  & $\pi(e)$ \\
\Rule{Extended Pattern}    & $\varpi$  & \is{}  & $\pi$ \Alt{} $c$ \Alt{} $x$ \\
\Rule{Pattern}             & $\pi$     & \is{}  & $\mu$ \Alt{} $\varrho$ \Alt{} $\eta$ \Alt{} $\chi$ \Alt{} $\varsigma$ \Alt{} $\_$ \\
\Rule{Constructor Pattern} & $\mu$     & \is{}  & \code{match<}$\tau\left[,l\right]$\code{>(}$\varpi^*$\code{)} \\
\Rule{Guard Pattern}       & $\varrho$ & \is{}  & $\pi \models \xi$ \\
\Rule{n+k Pattern}         & $\eta$    & \is{}  & $\xi$ \\
\Rule{Variable Pattern}    & $\chi^{\mathsf{variable}\langle\tau\rangle}$   \\
\Rule{Value Pattern}       & $\varsigma^{\mathsf{value}\langle\tau\rangle}$ \\
\Rule{Wildcard Pattern}    & $\_^{\mathsf{wildcard}}$                       \\
\Rule{Lazy Expression}     & $\xi$     & \is{}  & $\varsigma$ \Alt{} $\chi$ \Alt{} $\xi \oplus c$ \Alt{} $c \oplus \xi$ \Alt{} $\ominus \xi$ \Alt{} $(\xi)$ \Alt{} $\xi \oplus \xi$ \Alt{} $\varphi(\xi^*)$ \\
\Rule{Lazy Function}       & $\varphi^{\xi^*\rightarrow \xi}$ \\
\Rule{Unary Operator}      & $\ominus$ & $\in$  & $\lbrace*,\&,+,-,!,\sim\rbrace$ \\
\Rule{Binary Operator}     & $\oplus$  & $\in$  & $\lbrace*,/,\%,+,-,\ll,\gg,\&,\wedge,|,<,\leq,>,\geq,=,\neq,\&\&,||\rbrace$ \\
\Rule{Type-Id}             & $\tau$    &        & C++\cite[\textsection A.7]{C++11} \\
\Rule{Statement}           & $s$       &        & C++\cite[\textsection A.5]{C++11} \\
\Rule{Expression}          & $e^\tau$  &        & C++\cite[\textsection A.4]{C++11} \\
\Rule{Constant-Expression} & $c^\tau$  &        & C++\cite[\textsection A.4]{C++11} \\
\Rule{Identifier}          & $x^\tau$  &        & C++\cite[\textsection A.2]{C++11} \\
\end{tabular}
\caption{Abstract syntax of our pattern-matching SELL}
\label{syntax}
\end{figure}

Figure~\ref{syntax} presents the abstract syntax of our pattern-matching SELL. It presents 
the syntax embedded into the C++ without sacrificing 
the clarity of presentation. The idea is to show which interactions are possible 
within our SELL, while leaving the details of their implementation to 
\textsection\ref{sec:impl}. Where the specific technique we use to achieve such 
interactions crucially depends on the types of the entities involved,
we mention their type in the superscript. This dependence on the 
type system of the host language was also the reason why we chose abstract 
syntax over traditional EBNF. We make use 
of few non-terminals from the host language in order to put our constructs into 
context.

\emph{Match statement} is an analog of a switch statement with patterns as case 
clauses. Similar control structures exist in many programming languages and 
date back to at least Simula's Inspect statement~\cite{Simula67}.
In a library-based solution, we require it to be closed with a dedicated 
\code{EndMatch} macro to ensure proper nesting.

We support four kinds of \emph{case clauses}: \code{Qua}, \code{When}, 
\code{Case}, and \code{Otherwise}.
The distinction between them is only important for the library 
implementation and can trivially be inferred in a compiler solution.
A \code{Qua}-clause is the most general clause taking an  
expression that identifies the target type as well as a list of extended 
patterns.
A \code{Qua}-clause permits nested patterns, but requires all the 
variables used in the patterns to be explicitly pre-declared. \code{Case}-clause 
only accepts simple patterns, conveniently introducing all the variables into the 
clause's scope. 
A \code{When}-clause takes only patterns while its target type is 
the subject type.
An \code{Otherwise} clause is an irrefutable clause that is 
semantically equivalent to \code{Case}-clause with subject type used as a target 
type. When used it should be the last clause of the match statement.

A \emph{Target expression} is used by case clauses as either a target type or 
a constant value, representing \emph{layout}. \emph{Layout} is an enumerator 
that the user may use to define alternative bindings for the same class. They are 
discussed in \textsection\ref{sec:bnd}. The use of layout as target 
expression is only allowed for union encoding of algebraic data types 
(\textsection\ref{sec:unisyn}), in which case the library assumes the target 
type to be the subject type.

A \emph{Match expression} is an inline version of the match statement with 
a single \code{Qua}-clause. Applying a pattern to a subject checks whether the 
subject matches the pattern.

\emph{Pattern} summarizes all the patterns supported by the library. 
\emph{Extended pattern} indicates contexts in which our library implicitly 
permits the use constants as \emph{value patterns} and regular C++ variables as 
\emph{variable patterns}. The library recognizes them and transforms into 
$\varsigma$ and $\chi$ respectively.
A \emph{Constructor pattern} takes a target type, an optional layout and a list of 
nested sub-patterns.
\emph{Guard patterns} are composed from a pattern and a condition separated with 
\code{|=}.
We chose operator \code{|=} because of its low precedence 
allows most other operators to be used inside the 
condition without parenthesis. The right operand of \code{|=} is allowed to make use of any 
variables bound in the left operand. When used on arguments of a constructor 
pattern, it is also allowed to make use of any variables bound by preceding 
argument positions. 
\emph{n+k patterns} are a subset of \emph{lazy expressions} for which the user has 
provided \emph{solvers} -- overloaded functions defining semantics of matching a 
value against an expression(\textsection~\ref{sec:slv}).
\emph{Variable patterns} refers to variables whose C++ type is \code{variable<T>} for 
any given type \code{T}.
A \emph{Value pattern} is almost never declared explicitly, 
but is implicitly introduced by the library in the contexts where $c$ is 
accepted.
A \emph{Wildcard pattern} is represented with a constant of type 
\code{wildcard}.
A \emph{Lazy expression} refers to lazily evaluated expressions introduced by our SELL, 
as opposed to eagerly evaluated expressions, directly supported by C++. The use 
of $c$ indicates contexts in which constants can be used as lazy expressions and 
is similarly replaced with $\varsigma$. \emph{Lazy function} represents 
functions that can participate in lazy evaluations. Such functions have to be 
declared in certain way and are discussed in \textsection\ref{sec:slv}. 

\emph{Binary operator} and \emph{unary operator} name a subset of C++ operators we 
make use of and provide support for in our pattern-matching library. 
The remaining syntactic categories refer to non-terminals in the C++ grammar 
bearing the same name.

\subsection{Typing Rules}

\begin{figure}[h]
\begin{mathpar}

\inferrule[T-Var]
{}
{\Gamma\vdash \chi : \Variable{T}}

\inferrule[T-Value]
{}
{\Gamma\vdash \varsigma : \Value{T}}

\inferrule[T-Wildcard]
{}
{\Gamma\vdash \_ : \Wildcard}

\inferrule[T-Unary]
{\Gamma\vdash \xi : E}
{\Gamma\vdash \ominus \xi : \ExprU{F_\ominus}{E} }

\inferrule[T-Binary]
{\Gamma\vdash \xi_1 : E_1 \\ \Gamma\vdash \xi_2 : E_2}
{\Gamma\vdash \xi_1 \oplus \xi_2 : \ExprB{F_\oplus}{E_1}{E_2} }

%\inferrule[T-Binary-Const-Left]
%{\Gamma\vdash \xi : E \\ \Gamma\vdash c : T}
%{\Gamma\vdash c \oplus \xi : \ExprB{F_\oplus}{\Value{T}}{E} }

%\inferrule[T-Binary-Const-Right]
%{\Gamma\vdash \xi : E \\ \Gamma\vdash c : T}
%{\Gamma\vdash \xi \oplus c : \ExprB{F_\oplus}{E}{\Value{T}} }

\inferrule[T-Function]
{\Gamma\vdash \xi_1 : E_1 \\ \cdots \\ \Gamma\vdash \xi_k : E_k}
{\Gamma\vdash \varphi(\xi_1,\cdots,\xi_k) : \ExprK{F_\varphi}{E_1}{E_k} }

\inferrule[T-Guard]
{\Gamma\vdash \pi : E_1 \\ \Gamma\vdash \xi : E_2}
{\Gamma\vdash \pi \models \xi : \Guard{E_1}{E_2} }

\inferrule[T-Constructor]
{\Gamma\vdash \varpi_1 : E_1 \\ \cdots \\ \Gamma\vdash \varpi_k : E_k}
{\Gamma\vdash \mathsf{match}\langle T\left[,l\right]\rangle(\varpi_1,\cdots,\varpi_k) : \Cnstr{T\left[,l\right]}{E_1}{E_k} }

%\inferrule[T-Extended-Pattern]
%{ \varpi = \pi \\ \Gamma\vdash \pi : E}
%{\Gamma\vdash \varpi : E}

%\inferrule[T-Extended-Value]
%{ \varpi = c \\ \Gamma\vdash c : T}
%{\Gamma\vdash \varpi : \Value{T}}

%\inferrule[T-Extended-Var]
%{ \varpi = x \\ \Gamma\vdash x : T}
%{\Gamma\vdash \varpi : \Variable{T}}

\end{mathpar}
\caption{Typing rules for our pattern-matching SELL}
\label{typing}
\end{figure}

Figure~\ref{typing} shows rules we use to type expressions in our SELL. The 
types presented are not necessarily the exact C++ types we use to encode them, 
but we keep the correspondence as close as possible to reflect the actual 
implementation. We use the following type constructors, indicated with their 
arity: $\CWildcard^0$, $\CValue^1$, $\CVariable^1$, $\CExpr^{1+n}$, $\CGuard^2$, 
$\CCnstr^{1+n}$. We assume that type variables $T_i$ range over any C++ types, 
while $E_i$ only range over types marked with these type constructors, to which 
we refer as \emph{SELL-types}.

The judgments are of the traditional form $\Gamma\vdash \varpi : E$ that can be 
interpreted as given a typing environment $\Gamma$, an extended pattern $\varpi$ is 
given a SELL-type $E$. $\Gamma$ represents the typing context of the C++ 
compiler with the allowance for our simplified representation of SELL-types.
Types $F_\oplus$, $F_\ominus$ and $F_\varphi$ are described in greater details 
in \textsection\ref{sec:sem}, while for the purpose of typing they can be 
interpreted as types that uniquely identify operations $\oplus$, $\ominus$ and 
$\varphi$ respectively.

To avoid confusion we would like to point out that syntactic categories $\chi$, 
$\varsigma$ and $\_$ are defined as objects of C++ types \code{value<T>}, 
\code{variable<T>} and \code{wildcard}, while here we type them with SELL-types 
$\Value{T}$, $\Variable{T}$ and $\Wildcard$. Internally these types are the same 
of course.

\subsection{Semantics}
\label{sec:sem}

We use natural semantics\cite{Kahn87} to describe the semantics of our 
pattern-matching extension. Because our SELL can be customized in a number of 
ways, we make use of several semantic functors that let the user define the 
semantics of the following operations:

\begin{compactitem}
\setlength{\itemsep}{0pt}
\setlength{\parskip}{0pt}
\item Type casting: $F_{dc}(\tau,v)$
\item Lazily evaluated functions: $F_\oplus,F_\ominus,F_\varphi$
\item Structural decomposition: $\Delta_i^{\tau,l}$
\item Algebraic decomposition: $\mathsf{solve}(\eta,v)$
\end{compactitem}

\noindent
The type of the subject used in pattern matching is not always the same as the 
type that a given pattern expects. The library in such a case may need to 
perform type casting of the subject, which may involve but is not limited to 
down-casting, up-casting, cross-casting or conversion. Depending on the types 
involved, such casting can be performed in different ways, which is why we 
abstract from a concrete semantics of such an operation with functor $F_{dc}$. 
We use the notation $F_{dc}(\tau,v)$ to refer to the result of casting value $v$ 
to target type $\tau$, which may result in a dedicated value $\nullptr$ that 
indicates impossibility of such a cast. We discuss various implementations of 
such a functor in \textsection\ref{sec:unisyn}.

Every function $\varphi$ that the user would like to be able to call lazily 
requires definition of a functor $F_\varphi$ that defines the semantics of such 
operation on any given argument types. The library defines such semantic objects 
$F_\oplus$ and $F_\ominus$ for every binary operation $\oplus$ and unary 
$\ominus$ it supports. The user is responsible for defining semantic functor 
$F_\varphi$ for every function $\varphi$ she would like to be able to evaluate 
lazily or use in a generalized n+k pattern. We show how to define such functors 
in \textsection\ref{sec:impl}, while here we use the notation $F(v_1,\cdots,v_k)$ 
to refer to the value representing the result of applying such a functor to 
values $v_1,\cdots,v_k$.

Each variant of an algebraic data type in a functional language has exactly one 
constructor, which makes it ideally suitable for structural decomposition of the 
type with pattern matching. Classes in C++ are allowed to have multiple 
constructors, which is why we need a mechanism that would let the user specify 
structural decomposition of a class. We do this with the help of bindings 
(\textsection\ref{sec:bnd}) represented here with a functor $\Delta_i^{\tau,l}$. 
We use the notation $\Delta_i^{\tau,l}(v)$ to refer to the value representing 
the $i^{th}$ component in layout $l$ of the structural decomposition of a value 
$v$ of type $\tau$.

Lastly, we let the user define the exact meaning of matching a value $v$ against 
an expression $\eta$ by case analysis on the structure of $\eta$. The exact 
details of defining such algebraic decomposition are given in 
\textsection\ref{sec:slv}, while here we use the notation $\mathsf{solve}(\eta,v)$ 
to refer to a boolean value indicating whether the generalized n+k pattern 
$\eta$ was accepted (true) or rejected (false).

We model the run-time environment of our SELL as a map $\Sigma: \chi\rightarrow T$ 
since our variables $\chi$ either hold a value of type \code{T} or refer to another 
variable of that type. In addition to meta-variables we have seen already, 
meta-variables $u,v$ and $b^{bool}$ range over values.

%\subsubsection{Evaluation Rules}
%\label{sec:eval}

\begin{figure}[h]
\begin{mathpar}

\inferrule[E-Value]
{\varsigma = \Value{\tau}(v)}
{\varsigma \lazyevals v}

\inferrule[E-Var]
{\chi = \Variable{\tau}(v)}
{\chi \lazyevals v}

\inferrule[E-Unary]
{\xi \lazyevals v}
{\ominus \xi \lazyevals F_\ominus(v)}

\inferrule[E-Binary]
{\xi_1 \lazyevals v_1 \\ \xi_2 \lazyevals v_2}
{\xi_1 \oplus \xi_2 \lazyevals F_\oplus(v_1,v_2)}

%\inferrule[E-Binary-Const-Left]
%{\xi \lazyevals v}
%{\xi \oplus c \lazyevals F_\oplus(v,c)}

%\inferrule[E-Binary-Const-Right]
%{\xi \lazyevals v}
%{c \oplus \xi \lazyevals F_\oplus(c,v)}

\inferrule[E-Function]
{\xi_1 \lazyevals v_1 \\ \cdots \\ \xi_k \lazyevals v_k}
{\varphi(\xi_1,\cdots,\xi_k) \lazyevals F_\varphi(v_1,\cdots,v_k)}

\end{mathpar}
\caption{Evaluation rules}
\label{evaluation}
\end{figure}

Figure~\ref{evaluation} shows the evaluation rules used to evaluate lazy expressions 
that our SELL introduces. The judgments are of the form $\Sigma\vdash \xi \lazyevals v$ 
stating that lazy expression $\xi$ evaluates to a value $v$ in a run-time 
environment $\Sigma$. We do not mention the run-time environment in the rules 
for brevity since the evaluation does not modify it.

%\subsubsection{Semantics of Matching Expressions}
%\label{sec:semme}

\begin{figure}[h]
\begin{mathpar}
\inferrule[P-Wildcard]
{}
{\Sigma\vdash \_(v_e) \evals \True,\Sigma}

\inferrule[P-Value]
{\varsigma \lazyevals u}
{\Sigma\vdash \varsigma^\tau(v_e) \evals (u=v),\Sigma}

\inferrule[P-Variable]
{u=F_{dc}(\tau,v_e)}
{\Sigma\vdash \chi^{\tau}(v_e) \evals (u \neq \nullptr{}),\Sigma[\chi\leftarrow u]}

\inferrule[P-n+k-Pattern]
{\Sigma\vdash \mathsf{solve}(\xi,v_e) \evals v,\Sigma'}
{\Sigma\vdash \xi(v_e) \evals v,\Sigma'}

\inferrule[P-Guard]
{\Sigma\vdash \pi(v_e) \evals b_\pi,\Sigma' \\ \Sigma'\vdash \xi \lazyevals b_\xi}
{\Sigma\vdash (\pi \models \xi)(v_e) \evals (b_\pi \wedge b_\xi),\Sigma'}

\inferrule[P-Constructor-Nullptr]
{F_{dc}(\tau,v_e)=\nullptr{}}
{\Sigma\vdash (\mathsf{match}\langle\tau\left[,l\right]\rangle(\varpi_1,...,\varpi_k))(v_e) \evals \False,\Sigma}

\inferrule[P-Constructor-Reject]
{ u=F_{dc}(\tau,v_e) \\
 \Sigma_1    \vdash \varpi_1(\Delta_1^{\tau,l}(u))         \evals \True, \Sigma_2 \\ \cdots \\
 \Sigma_{i-1}\vdash \varpi_{i-1}(\Delta_{i-1}^{\tau,l}(u)) \evals \True, \Sigma_i \\
 \Sigma_i    \vdash \varpi_i(\Delta_i^{\tau,l}(u))         \evals \False,\Sigma_{i+1}
}
{\Sigma\vdash (\mathsf{match}\langle\tau\left[,l\right]\rangle(\varpi_1,...,\varpi_k))(v_e) \evals \False,\Sigma_{i+1}}

\inferrule[P-Constructor-Accept]
{ u=F_{dc}(\tau,v_e) \\
 \Sigma_1    \vdash \varpi_1(\Delta_1^{\tau,l}(u)) \evals \True, \Sigma_2 \\ \cdots \\
 \Sigma_k    \vdash \varpi_k(\Delta_k^{\tau,l}(u)) \evals \True, \Sigma_{k+1}
}
{\Sigma\vdash (\mathsf{match}\langle\tau\left[,l\right]\rangle(\varpi_1,...,\varpi_k))(v_e) \evals \True,\Sigma_{k+1}}

\end{mathpar}
\caption{Semantics of matching expressions}
\label{exprsem}
\end{figure}

The rule set in Figure~\ref{exprsem} deals with pattern application $\pi(e)$, 
which essentially performs matching of a pattern $\pi$ against an expression 
$e$. To avoid dealing with the C++ semantics, we assume that the expression $e$ 
has already been evaluated to a value $v_e$. Our judgments are thus of the 
form $\Sigma\vdash \pi(v_e) \evals v,\Sigma'$ that can be interpreted as 
following: given an environment $\Sigma$ and a value $v_e$ representing the 
result of evaluating subject expression $e$ according to the C++ semantics, 
pattern application $\pi(v_e)$ results in value $v$ and environment $\Sigma'$. 

%\subsubsection{Semantics of Match Statement}
%\label{sec:semms}

\begin{figure}[h]
\begin{mathpar}
\inferrule[Match-True]
{ v_e \neq \nullptr \\
 \Sigma      \vdash_{v_e} C_1    \evals \False,\Sigma_1     \\ \cdots \\
 \Sigma_{i-2}\vdash_{v_e} C_{i-1}\evals \False,\Sigma_{i-1} \\
 \Sigma_{i-1}\vdash_{v_e} C_i    \evals \True, \Sigma_i
}
{\Sigma\vdash \mathsf{Match}(v_e) \{ \left[C_i \vec{s}_i\right]^*_{i=1..n} \} \mathsf{EndMatch} \evals i,\Sigma_i}

\inferrule[Match-False]
{ v_e \neq \nullptr \\
 \Sigma      \vdash_{v_e} C_1    \evals \False,\Sigma_1 \\ \cdots \\
 \Sigma_{n-1}\vdash_{v_e} C_n    \evals \False,\Sigma_n
}
{\Sigma\vdash \mathsf{Match}(e) \{ \left[C_i \vec{s}_i\right]^*_{i=1..n} \} \mathsf{EndMatch} \evals 0,\Sigma_n}

\inferrule[Qua]
{\Sigma \vdash \mathsf{match}\langle\tau,l\rangle(\vec{\varpi})(v_e) \evals b,\Sigma' }
{\Sigma \vdash_{v_e} \mathsf{Qua}(\tau,\vec{\varpi}) \evals b,\Sigma'[\mathsf{matched}^\tau\rightarrow F_{dc}(\tau,v_e)]}

\inferrule[When]
{\Sigma \vdash_{v^\tau_e} \mathsf{Qua}(\tau,\vec{\varpi}) \evals b,\Sigma'}
{\Sigma \vdash_{v^\tau_e}     \mathsf{When}(\vec{\varpi}) \evals b,\Sigma'}

\inferrule[Case]
{\Delta_i^\tau : \tau \rightarrow \tau_i, i=1..k \\
 \Sigma[x_i^{\tau_i}\rightarrow\nullptr]_{i=1..k} \vdash_{v_e} \mathsf{Qua}(\tau,x_1,...,x_k) \evals u,\Sigma' }
{\Sigma \vdash_{v_e} \mathsf{Case}(\tau,x_1,...,x_k) \evals u,\Sigma'}

\inferrule[Otherwise]
{\Sigma \vdash_{v^\tau_e} \mathsf{Case}(\tau,\vec{x}) \evals u,\Sigma'}
{\Sigma \vdash_{v^\tau_e} \mathsf{Otherwise}(\vec{x}) \evals u,\Sigma'}
\end{mathpar}
\caption{Semantics of match-statement}
\label{stmtsem}
\end{figure}

The rule set in Figure~\ref{stmtsem} describes the semantics of a \emph{match 
statement}. In order to avoid dealing with the semantics of the C++ statements, 
we define the semantics of a match-statement to be the index of the matching case 
clause and the run-time environment right before the clause's statement, or $0$ 
if none of the clauses matched. The judgments are thus of the form 
$\Sigma\vdash M \evals v,\Sigma'$ for match statement, and are slightly extended 
for case clauses $\Sigma\vdash_{v_e} C \evals b,\Sigma'$ with value $v_e$ of a 
subject that is passed along from the match statement onto the clauses.

The rules essentially describe the first-fit strategy for evaluating the clauses.
Evaluation of a \code{Qua}-clause is equivalent to evaluation of a corresponding 
match-expression on a constructor pattern. Successful match will introduce into 
the local scope of the clause a variable \code{matched} bound to the subject 
properly casted to the target type $\tau$. Evaluation of \code{When}-clause 
amounts to evaluation of a corresponding \code{Qua}-clause with target type 
being the subject type. Evaluation of \code{Case}-clauses amounts to evaluation 
of \code{Qua}-clauses in the environment extended with variables passed as 
arguments to the clause. Evaluation of default clause amounts to evaluating a 
corresponding \code{Case}-clause with target type being the subject type.

