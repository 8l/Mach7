\section{Implementation} %%%%%%%%%%%%%%%%%%%%%%%%%%%%%%%%%%%%%%%%%%%%%%%%%%%%%%%
\label{sec:impl}

The \code{Match}-statement has been implemented so as to be as efficient as or 
more efficient than conventional object-oriented and union-based 
alternatives\cite{TypeSwitch}. Our implementation of patterns and lazy 
evaluation of expressions is based on \emph{Expression Templates} 
\cite{Veldhuizen95expressiontemplates,vandevoorde2003c++}. It encodes 
expression trees with types, which are hidden from the user through the use of 
overloading.

There are 6 kinds of expression templates in our library: \code{wildcard}, 
\code{value<T>}, \code{variable<T>}, \code{expr<F,E...>}, \code{guard<E1,E2>}, 
\code{ctor<T,E...>}. Each models a \code{Pattern} concept.
Using the notation from C++0x\cite{C++0xConcepts}:

\begin{lstlisting}[keepspaces,columns=flexible]
concept Pattern<typename T> 
{
    typename R; // the return type of () isconvertible to bool
    Convertible<R,bool>;
    template <typename U> R operator()(const U& u) const; // application operator
};
\end{lstlisting}

Each of our six pattern kinds 
implements the application operator according to the semantics presented in 
Figure~\ref{exprsem}. The application operator's result has to be 
convertible to bool;
\code{true} indicates a successful match. A class might have several overloads of 
the above operator that distinguish cases of interest. We summarize the requirements on template parameters of each of our 
pattern in Figure~\ref{xt-reqs}.

\begin{figure}[h]
\centering
\begin{tabular}{llll}
{\bf Pattern}       & {\bf Parameters}          & {\bf Argument of application operator U}         \\ \hline
\code{wildcard}     & --                        & --                                               \\
\code{value<T>}     & \code{Regular<T>}         & \code{Convertible<U,T>}                          \\
\code{variable<T>}  & \code{Regular<T>}         & \code{Convertible<U,T>}                          \\
\code{expr<F,E...>} & \code{LazyExpression<E>}  & \code{Convertible<U,expr<F,E...>::result_type>}  \\
\code{guard<E1,E2>} & \code{LazyExpression<Ei>} & any type accepted by \code{E1::operator()}       \\
\code{ctor<T,E...>} & \code{Polymorphic<T>}     & \code{Polymorphic<U>} for open encoding          \\
                    & \code{Object<T>}          & \code{is_base_and_derived<U,T>} for tag encoding \\
\end{tabular}
\caption{Requirements on parameters and argument type of an application operator}
\label{xt-reqs}
\end{figure}

To support lazily evaluated expressions, classes \code{value<T>}, 
\code{variable<T>} and \code{expr<F,E...>} also model a \code{LazyExpression} 
concept:

\begin{lstlisting}[keepspaces,columns=flexible]
concept LazyExpression<typename T> 
{
    typename result_type;
    operator result_type(const T&); // conversion operator (to result_type)
};
\end{lstlisting}

\noindent
The \code{result_type} defines the type of the result of an argument expression.
It is \code{T} for \code{value<T>} and \code{variable<T>}. 
For \code{expr<F,E...>} it is defined to be \code{decltype(F()(E...))}, 
indicating the result of applying \code{F} to arguments of type 
\code{E...} .
The conversion operator invokes the evaluation of the 
lazy expression. The evaluation is performed according to the semantics 
presented in Figure~\ref{evaluation}.

Concepts were not included in C++11, so we emulate them using overloading and 
\code{enable_if}~\cite{jarvi:03:cuj_arbitrary_overloading}.

Each (possibly overloaded) operator and function that can be evaluated lazily 
and matched against in generalized n+k patterns are represented with a class whose 
only purpose is to transparently forward the call to an overloaded function that 
implements the semantics of the operation. We refer to such a class together with 
the overloaded function it forwards the calls to as a \emph{semantic functor}.
For example, this functor defines semantics of multiplication:

\begin{lstlisting}[keepspaces,columns=flexible]
struct mult 
{
  template <class A, class B> 
  auto operator()(A&& a, B&& b) const -> decltype(a*b) 
  { return std::forward<A>(a) * std::forward<B>(b); }   
};
\end{lstlisting}

\noindent
Unlike similar classes in STL, our representation of operations is not 
parameterized with argument types. This simplifies defining overloads of 
\code{solve} as shown below, reflecting the structure of the 
expression.

Variables of types \code{variable<T>} and \code{wildcard} as well as values 
wrapped into \code{value<T>} (implicitly by the library or explicitly with a 
call to function \code{val()}) constitute simple expressions in our pattern 
sub-language. Complex expressions are built by applying C++ operators 
listed in Figure~\ref{syntax} and functions overloaded on our lazy expressions. 
To support that, for every unary operator $\ominus$ and binary operator $\oplus$ 
and the semantic functors $F_\ominus$ and $F_\oplus$ we define for them, the 
library introduces:

\begin{lstlisting}[keepspaces]
template <LazyExpression E1>
    expr<@$F_\ominus$@,E1> operator@$\ominus$@(E1&& e1) noexcept 
    { return expr<@$F_\ominus$@,E1>(std::forward<E1>(e1)); }
template <LazyExpression E1, LazyExpression E2>
    expr<@$F_\oplus$@,E1,E2> operator@$\oplus$@(E1&& e1, E2&& e2) noexcept 
    { return expr<@$F_\oplus$@,E1,E2>(std::forward<E1>(e1),std::forward<E2>(e2)); }
template <Pattern P1, LazyExpression E2>
    guard<P1,E2> operator@$\models$@(P1&& p1, E2&& e2) noexcept 
    { return guard<P1,E2>(std::forward<P1>(p1),std::forward<E2>(e2)); }
template <typename T, LazyExpression... E>
    ctor<T,E...> match(E&&... e) noexcept 
    { return ctor<T,E...>(std::forward<E>(e)...); }
\end{lstlisting}

\subsection{Structural Decomposition}
\label{sec:bnd}

We use compile-time reflection to let the user specify information about 
a class hierarchy to the library. The information is provided as a specialization of a 
class-template \emph{bindings}. Among other things, bindings let the 
user define the semantic functor $\Delta_i^{\tau,l}$ we introduced in 
\textsection\ref{sec:sem}. The grammar in Figure~\ref{bind-syntax} defines the 
entities that may constitute a binding definition.

\begin{figure}[h]
\centering
\begin{tabular}{lp{1em}cl}
\Rule{bindings}                &           & \is{}  & $\delta^*$ \\
\Rule{binding definition}      & $\delta$  & \is{}  & \code{template <}$\left[\vec{p}\right]$\code{>} \\
                               &           &        & \code{struct bindings<} $\tau[$\code{<}$\vec{p}$\code{>}$]\left[,l\right]$\code{>} \\
                               &           &        & \code{\{} $\left[ks\right]\left[kv\right]\left[bc\right]\left[cm^*\right]$ \code{\};} \\
\Rule{class member}            & $cm$      & \is{}  & \code{CM(}$c^{size\_t},q$\code{);} \\
\Rule{kind selector}           & $ks$      & \is{}  & \code{KS(}$q$\code{);}    \\
\Rule{kind value}              & $kv$      & \is{}  & \code{KV(}$c$\code{);}    \\
\Rule{base class}              & $bc$      & \is{}  & \code{BC(}$\tau$\code{);} \\
\Rule{template-parameter-list} & $\vec{p}$ &        & C++\cite[\textsection A.12]{C++11} \\
\Rule{qualified-id}            & $q$       &        & C++\cite[\textsection A.4]{C++11} \\
\end{tabular}
\caption{Syntax for defining bindings}
\label{bind-syntax}
\end{figure}

\noindent
Any type $\tau$ may have arbitrary number of \emph{bindings} associated with it 
and distinguished through the \emph{layout} parameter $l$. The \emph{default 
binding} which omits layout parameter is implicitly associated with the layout whose
value is equal to predefined constant \code{default_layout = size_t(}$\sim$\code{0)}. 
A \emph{Binding definition} is a specialization of
\code{template <typename T, size_t l = default_layout> struct bindings;}
The body of the class consists of a sequence of specifiers, which generate the 
necessary definitions for querying bindings by the library code. Note that 
binding definitions made this way are \emph{non-intrusive} since the original 
class definition is not touched. They also respect \emph{encapsulation} since 
only the public members of the target type will be accessible from within 
\code{bindings} specialization.

A \emph{Class Member} specifier \code{CM(}$c,q$\code{)} takes a (zero-based) binding 
position $c$ and a member $q$, whose value will be matched against in $\tau$'s 
construction pattern. Qualified identifier is allowed to be of one of the 
following kinds:

\begin{compactitem}
\setlength{\itemsep}{0pt}
\setlength{\parskip}{0pt}
\item Data member of the target type
\item Nullary member-function of the target type
\item Unary external function taking the target type by pointer, reference or value.
\end{compactitem}

\noindent
Using \code{CM} specifier a user defines the semantic functor 
$\Delta_i^{\tau,l},i=1..k$ we introduced in \textsection\ref{sec:sem} as 
following:

\begin{lstlisting}[keepspaces]
template <typename... T> struct bindings<@$\tau$@<T...>> 
    { CM(0, @$\tau$@<T...>::member@$_0$@); ... CM(@$k$@, @$\tau$@<T...>::member@$_k$@); };
\end{lstlisting}

\noindent
A \emph{Kind Selector} specifier \code{KS(}$q$\code{)} is used to specify a member 
of the subject type that will uniquely identify the variant for \emph{tagged} 
and \emph{union} encodings. The member $q$ can be of any of the three categories 
listed for \code{CM}, but is required to return an \emph{integral type}.
A \emph{Kind Value} specifier \code{KV(}$c$\code{)} is used by \emph{tagged} and 
\emph{union} encodings to specify a constant $c$ that uniquely identifies given 
variant. 
A \emph{Base Class} specifier \code{BC(}$\tau$\code{)} is used by the \emph{tagged}
encoding to specify an immediate base class of the class whose bindings we 
define.

A \emph{Layout} parameter $l$ can be used to define multiple bindings for the same 
target type. This is particularly essential for \emph{union} encoding where the 
types of the variants are the same as the type of subject and thus layouts 
become the only way to associate variants with position bindings. For this 
reason, we require binding definitions for \emph{union} encoding always use the 
same constant $l$ as a kind value specified with \code{KV(l)} and the layout 
parameter $l$!

\subsection{Algebraic Decomposition}
\label{sec:slv}

Traditional approaches to generalizing n+k patterns treat matching a pattern 
$f(x,y)$ against a value $v$ as solving an equation $f(x,y)=v$\cite{OosterhofThesis}. 
Such an interpretation is well defined when there are zero or one solutions,
but alternative interpretations are possible when there are multiple solutions. 
We handle n+k patterns by 
associating the mathematical notation with the mathematical objects it 
represents. 
We associate elements of such notation with 
sub-components of the matched mathematical entity, which 
effectively lets us decompose it into parts. The structure of the expression 
tree used in the notion is an analog of a constructor symbol in structural 
decomposition, while its leaves are placeholders for parameters to be matched 
against or inferred from the mathematical object in question. In essence,
algebraic decomposition is to mathematical objects what views are to algebraic 
data types. We demonstrate with examples.

\begin{compactitem}
\setlength{\itemsep}{0pt}
\setlength{\parskip}{0pt}
\item An expression $n/m$ is often used to decompose a rational number into 
      numerator and denominator.
\item Euler notation $a+bi$ with $i$ being an imaginary unit is used to 
      decompose a complex number into real and imaginary parts. Similarly, 
      expressions $r(cos \phi + i\mathrm{sin} \phi)$ and $re^{i\phi}$ are used to 
      decompose it into polar form.
\item An object representing 2D line can be decomposed with slope-intercept form 
      $mX+c$, linear equation form $aX+bY=c$ or two-points form 
      $(Y-y_0)(x_1-x_0)=(y_1-y_0)(X-x_0)$.
\item An object representing polynomial can be decomposed for a specific degree: 
      $a_0$, $a_1X^1+a_0$, $a_2X^2+a_1X^1+a_0$ etc.
\item An element of a vector space can be decomposed along some sub-spaces of 
      interest. For example a 2D vector can be matched against $(0,0)$, $aX$, 
      $bY$, $aX+bY$ to separate the general case from those when one or both
      components of vector are $0$.
\end{compactitem}

\noindent
Expressions $i$, $X$ and $Y$ in these examples are not variables, but named 
constants of some dedicated type that lets the expression be generically 
decomposed into orthogonal parts. Note also that linear equation and two-point 
form for decomposing lines already include an equality sign, which makes it 
hard to give them semantics in an equational approach. It turns out that the 
equational approach can be generically expressed in our framework for many 
interesting cases of interest.

Applying equational approach to floating point arithmetic creates even more 
problems. Even when the solution is unique, it may not be representable by 
a given floating-point type and thus not satisfy the equation. Once we settle 
for an approximation, we open ourselves to even more decompositions that become 
possible with our approach.

\begin{compactitem}
\setlength{\itemsep}{0pt}
\setlength{\parskip}{0pt}
\item Matching $n/m$ with integer variables $n$ and $m$ against a floating point 
      value can be given semantics of finding the closest fraction to the 
      value.
\item Matching an object representing sampling of some random variable against
      expressions like $Gaussian(\mu,\sigma^2)$, $Poisson(\lambda)$ or 
      $Binomial(n,p)$ can be seen as distribution fitting. 
\item Any curve fitting in this sense becomes an application of pattern 
      matching. Precision in this case can be a global constant or explicitly 
      passed parameter of the matching expression.
\end{compactitem}

%\noindent
%We can make several observations from these examples:

%\begin{compactitem}
%\setlength{\itemsep}{0pt}
%\setlength{\parskip}{0pt}
%\item We might need to have the entire expression available to us in order to 
%      decompose its parts.
%\item Matching the same expression can have different meanings depending on 
%      types of objects composing the expression and the expected result. 
%\item An algorithm to decompose a given expression may depend on the types of 
%      objects in it and the type of the result. 
%\end{compactitem}

%\subsubsection{Solvers}

\noindent
The user of our library defines the semantics of decomposing a value of a given 
type \code{S} against an expression of shape \code{E} by overloading a function: 
\\ \code{template <LazyExpression E, typename S> bool solve(const E&, const S&);}

The first argument of the function takes an expression template representing an 
expression we are matching against, while the second argument represents the 
expected result. The following example defines a generic solver for 
multiplication by a constant:

\begin{lstlisting}[keepspaces]
template <LazyExpression E, typename T> requires Field<E::result_type>
bool solve(const expr<mult,E,value<T>>& e, const E::result_type& r) {
    return solve(e.m_e1,r/eval(e.m_e2));
}
@\halfline@
template <LazyExpression E, typename T> requires Integral<E::result_type>
bool solve(const expr<mult,E,value<T>>& e, const E::result_type& r) {
    T t = eval(e.m_e2);
    return r%t == 0 && solve(e.m_e1,r/t);
}
\end{lstlisting}

\noindent
The first overload is only applicable when the type of the result of the 
sub-expression models the \code{Field} concept. In this case, we can rely on the presence 
of a unique inverse and simply call division without any additional checks. The 
second overload uses integer division, which does not guarantee the unique 
inverse, and thus we have to verify that the result is divisible by the constant 
first. This last overload combined with a similar solver for addition of 
integral types is everything the library needs to properly handle the definition 
of the \code{fib} function from \textsection\ref{sec:syn}. It also demonstrates 
how an equational approach can be generically implemented for a number of 
expressions.

A generic solver capable of decomposing a complex value using the Euler 
notation is very easy to define by fixing the structure of expression:

\begin{lstlisting}[keepspaces]
template <LazyExpression E1, LazyExpression E2> 
    requires SameType<E1::result_type,E2::result_type>
bool solve(
        const expr<plus,expr<mult,E1,value<complex<E1::result_type>>>,E2>& e, 
        const complex<E1::result_type>& r);
\end{lstlisting}

\noindent
Note that the template facilities of C++ resemble pattern-matching facilities of 
other languages. We essentially use these compile-time patterns to describe the 
structure of the expression this solver is applicable to: $e_1*c+e_2$ with types 
of $e_1$ and $e_2$ being the same as type on which a complex value $c$ is 
defined. The actual value of the complex constant $c$ will not be known until 
run-time, but assuming its imaginary part is not $0$, we will be able to 
generically obtain the values for sub-expressions.

Our approach is largely possible due to the fact that the library only serves as 
an interface between expressions and functions defining their semantics and 
algebraic decomposition. The fact that the user explicitly defines the variables 
she would like to use in patterns is also a key as it lets us specialize not 
only on the structure of the expression, but also on the types involved. 
Inference of such types in functional languages would be hard or impossible as the 
expression may have entirely different semantics depending on the types of 
arguments involved. Concept-based overloading simplifies significantly the case 
analysis on the properties of types, making the solvers generic and composable.
The approach is also viable as expressions are decomposed at compile-time and 
not at run-time, letting the compiler inline the entire composition of solvers. 

An obvious disadvantage of this approach is that the more complex expression 
becomes, the more overloads the user will have to provide to cover all 
expressions of interest. The set of overloads will also have to be made 
unambiguous for any given expression, which may be challenging for novices. An 
important restriction of this approach is its inability to detect multiple uses 
of the same variable in an expression at compile time. This happens because 
expression templates remember the form of an expression in a type, so use of two 
variables of the same type is indistinguishable from the use of the same 
variable twice. This can be worked around by giving different variables 
(slightly) different types or making additional checks as to the structure of 
expression at run-time, but that will make the library even more verbose or 
incur a significant run-time overhead.

\subsection{Views}
\label{sec:view}

Support of multiple bindings through layouts in our library effectively enables 
a facility similar to Wadler's \emph{views}\cite{Wadler87}. The example from 
\textsection\ref{sec:bg} can be recoded in our SELL:

\begin{lstlisting}[keepspaces,columns=flexible]
enum { cartesian = default_layout, polar }; // Layouts
@\halfline@
template <typename T> struct bindings<std::complex<T>>
  { CM(0,std::real<T>); CM(1,std::imag<T>); };
template <typename T> struct bindings<std::complex<T>, polar>
  { CM(0,std::abs<T>);  CM(1,std::arg<T>); };
@\halfline@
template <typename T> using Cartesian = view<std::complex<T>>;
template <typename T> using Polar     = view<std::complex<T>, polar>;
@\halfline@
  std::complex<double> c; double a,b,r,f;
  if (match<std::complex<double>>(a,b)(c)) // default
  if (match<   Cartesian<double>>(a,b)(c)) // same as above
  if (match<       Polar<double>>(r,f)(c)) // view
\end{lstlisting}

\noindent
The C++ standard effectively enforces the standard library to use cartesian 
representation\cite[\textsection26.4-4]{C++11}, which is why we choose the 
\code{cartesian} layout to be the default. We then define bindings for each 
layout and introduce template aliases (an analog of typedefs for parameterized 
classes) for each of the views. Library class \code{view<T,l>} binds together a 
target type with one of its layouts, which can be used everywhere where an 
original target type was expected.

The important difference from Wadler's solution is that our views can only be 
used in a match expression and not as a constructor or arguments of a function 
etc.

\subsection{Unified Syntax}
\label{sec:unisyn}

C++ does not have direct support for algebraic data types, but they can be 
emulated in a number of ways. A general pattern-matching solution must be able 
to elegantly and efficiently deal with all of them. Consider an ML data type: 

\begin{lstlisting}[language=ML,escapechar=@]
datatype DT = @$C_1$@of{@$L_{11}:T_{11},...,L_{1m}:T_{1m}$@}|...|@$C_k$@of{@$L_{k1}:T_{k1},...,L_{kn}:T_{kn}$@}
\end{lstlisting}

\noindent There are at least 3 different ways to represent it in C++. Following 
Emir, we will refer to them as \emph{encodings}~\cite{EmirThesis}:

\begin{compactitem}
\setlength{\itemsep}{0pt}
\setlength{\parskip}{0pt}
\item Polymorphic Base Class (or \emph{polymorphic encoding} for short)
\item Tagged Class (or \emph{tagged encoding} for short)
\item Discriminated Union (or \emph{union encoding} for short)
\end{compactitem}

\noindent
In polymorphic and tagged encodings, base class \code{DT} represents algebraic 
data type, while derived classes represent variants. The only difference between 
the two is that in polymorphic encoding the base class has virtual functions, while 
in tagged encoding it has a dedicated member of integral type  (a ``type field'')  that uniquely 
identifies the variant -- derived class. 

\begin{lstlisting}[keepspaces,columns=flexible]
struct DT { virtual @$\sim$@DT{} };                 // polymorhpic
struct DT { enum kinds {@$c_1, ..., c_k$@} m_kind; }; // tagged
struct @$C_1$@ : DT{@$T_{11} L_{11};...T_{1m} L_{1m};$@} @$...$@ struct @$C_k$@ : DT{@$T_{k1} L_{k1};...T_{kn} L_{kn};$@} 
\end{lstlisting}

\noindent
Union encoding is similar to tagged encoding in using a dedicated member to 
distinguish the variant, but the variants are encoded as a union:

\begin{lstlisting}[keepspaces,columns=flexible]
struct DT {
  enum kinds {@$c_1, \cdots, c_k$@} m_kind;
  union {struct @$C_1\{T_{11} L_{11};...T_{1m} L_{1m};\} C_1;...$@struct @$C_k\{T_{k1} L_{k1};...T_{kn} L_{kn};\} C_k;$@};
};
\end{lstlisting}

\noindent
To uncover the actual variant in a polymorphic encoding, the user might use 
\code{dynamic_cast} (an approach used by ROSE\cite{SQ03}) or employ a visitor 
design pattern (an approach used by 
the Pivot\cite{Pivot09} and Phoenix\cite{Phoenix}). In case of tagged and union 
encodings, the user might use a simple switch statement to uncover the variant 
(approaches used by Clang\cite{Clang} and EDG\cite{EDG} respectively, as well as 
many others).

Polymorphic encoding is inherently \emph{open} as it can also be used to encode 
extensible and hierarchical datatypes. It is also type safe. Union encoding is inherently 
\emph{closed} as it cannot be used for either. Tagged encoding is suitable for 
extensible datatypes, but without additional support it is not useful for 
encoding hierarchical datatypes since it does not provide a way of checking 
whether two variants with given tags are disjoint.
Tagged and union encodings are not inherently type safe in the hands of 
programmers, but they are when a compiler generates them.

The syntactic structure that supports our match statement is generated by the 
C++ preprocessor from macros \code{Match}, \code{Case}, \code{Qua}, \code{When}, 
\code{Otherwise} and \code{EndMatch}. This happens before type analysis,
so we must make 
decisions based on a position of an argument in a macro, the number of arguments 
in a macro or postpone the decision untill later by generating some compile-time 
code that will be evaluated by C++ compiler or a run-time code executed by the 
program. Once a compiler is invoked, we can specialize the code based on 
properties of the types involved: e.g. subject type is polymorphic, the binding 
on subject type defines kind selector with \code{KS} etc. Using this information, 
we further specialize a class template that provides the necessary functions and 
meta programs to generate the support of an encoding used by the user in the 
most optimal way.

The unified syntax allows automatic 
redundancy checking (\textsection\ref{sec:bg}) on case clauses of a match statement.
C++ catch-handlers use first-fit 
semantics in handling exceptions, so we turn an entire match statement into a 
try-catch block with handlers accepting the target types. 
The compiler then warns if a more general catch handler precedes a more 
specific one, effectively performing redundancy checking for us.

%\subsection{Qualitative Comparison}
%\label{sec:qualcmp}
%
%For this experiment we have reimplemented a visitor based C++ pretty printer for 
%Pivot\cite{Pivot09} using our pattern-matching library. Pivot's class hierarchy 
%consists of 154 node kinds representing various entities in the C++ program. The 
%original code had 8 visitor classes each handling 5, 7, 8, 10, 15, 17, 30 and 63 
%cases, which we turned into 8 match statements with corresponding numbers of 
%case clauses. Most of the rewrite was performed by sed-like replaces that 
%converted visit methods into respective case-clauses. In several cases we had to 
%manually reorder case-clauses to avoid redundancy due to the first-fit semantics 
%of our match statement and automatic redundancy checking was invaluable for this 
%refactoring.
%
%During this refactoring we have made several simplifications that became obvious 
%in pattern-matching code, but were not in visitors code because of control 
%inversion. Simplifications that were applicable to visitors code were eventually 
%integrated into visitors code as well to make sure we do not compare 
%drastically different code. In any case we were making sure that both 
%approaches regardless of simplifications were producing byte-to-byte the same 
%output as the original pretty printer we started from.
%
%The size of executable for pattern-matching approach was smaller than that for 
%visitors. So was the source code. 
%
%%Listing parameter for a case clause always causes access to member. Best hope is 
%%that compiler will eliminate it if it is not needed. At the moment we do not 
%%have means to detect empty macro arguments or \_.
%
%In general from our rewriting experience we will not recommend rewriting 
%existing visitor code with pattern matching for the simple reason that pattern 
%matching code will likely follow the structure already set by the visitors. 
%Pattern matching was most effective when writing new code, where we could design 
%the structure of the code having the pattern-matching facility in our toolbox.
