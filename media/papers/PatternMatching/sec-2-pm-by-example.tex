\section{Pattern Matching by Example} %%%%%%%%%%%%%%%%%%%%%%%%%%%%%%%%%%%%%%%%%%
\label{sec:bg}

Pattern matching is an abstraction mechanism that provides syntax very close to 
mathematical notations. It allows the user tersely describe a (possibly 
infinite) set of values accepted by the pattern. A \emph{pattern} represents a 
predicate on values, and is usually more concise and readable than equivalent 
imperative code. Pattern matching is closely related to \emph{algebraic data 
types} and \emph{equational reasoning}. 
In ML and Haskell an \emph{Algebraic Data Type} is a data type 
each of whose values are picked from a disjoint sum of (possibly recursive) data 
types, called \emph{variants}. Each variant is marked with a unique 
symbolic constant called a \emph{constructor}. Constructors provide a convenient 
way of creating a value of its variant type as well as discriminating among 
variants through pattern matching. Consider a simple expression language:
\begin{center}
$exp$ \is{} $val$ \Alt{} $exp+exp$ \Alt{} $exp-exp$ \Alt{} $exp*exp$ \Alt{} $exp/exp$
\end{center}

\noindent
This can be represented in OCaml as an algebraic data type:

\begin{lstlisting}[language=Caml,keepspaces,columns=flexible]
type expr = Value  of int 
          | Plus   of expr * expr 
          | Minus  of expr * expr 
          | Times  of expr * expr 
          | Divide of expr * expr ;;
\end{lstlisting}

\noindent
The set of values described by such an algebraic data type is defined 
inductively as the least set closed under constructor functions of its variants.
A simple evaluator of such expressions can be defined:

\begin{lstlisting}[language=Caml,keepspaces,columns=flexible]
let rec eval e = match e with 
            Value  v     -> v 
          | Plus  (a, b) -> (eval a) + (eval b) 
          | Minus (a, b) -> (eval a) - (eval b)
          | Times (a, b) -> (eval a) * (eval b) 
          | Divide(a, b) -> (eval a) / (eval b) ;;
\end{lstlisting}

\noindent
There are two critical differences between algebraic data types and classes in 
object-oriented languages: definition of an algebraic data type is \emph{closed}, 
while its variants are \emph{disjoint}. Closedness means that once we have listed 
all the variants a given algebraic data type may have we cannot extend it with 
new variants without modifying its definition. Disjointedness means that a value 
of an algebraic data type belongs to exactly one of its variants. Neither is 
the case in object-oriented languages. Classes are \emph{extensible},
since new variants can be added through subclassing, as well as 
\emph{hierarchical}, since variants are not necessarily disjoint and can form 
subtyping relation between themselves. The above algebraic data type can be 
encoded in C++ as following:

\begin{lstlisting}[columns=flexible]
struct Expr { virtual @$\sim$@Expr(){}};
struct Value  : Expr { int value; };
struct Plus   : Expr { Expr* e1; Expr* e2; }; ...
struct Divide : Expr { Expr* e1; Expr* e2; };
\end{lstlisting}

\noindent
while a similar evaluator in our SELL is almost as terse as OCaml code:

\begin{lstlisting}[columns=flexible]
int eval(const Expr* e)
{
  Match(e) {
    Case(Value,  n)    return n;
    Case(Plus,   a, b) return eval(a) + eval(b); 
    Case(Minus,  a, b) return eval(a) - eval(b);
    Case(Times,  a, b) return eval(a) * eval(b); 
    Case(Divide, a, b) return eval(a) / eval(b);
  } EndMatch
}
\end{lstlisting}

\noindent
An expression \code{e} used as an argument of a \emph{matching expression} or a 
\emph{match statement}, is usually referred to as \emph{subject}, and its type 
as a \emph{subject type}. We will also refer to the type of the value expected 
by a given pattern as a \emph{target type}.

\emph{Polymorphic variants} in OCaml\cite{garrigue-98} and \emph{open data 
types} in Haskell\cite{LohHinze2006} allow addition of new variants later. 
These extensions are simpler however as they maintain the 
disjointedness property: open data types do not introduce any subtyping relation, 
while the subtyping relation on polymorphic variants relates anonymous 
combinations of variants and not the variants themselves. In contrast, 
the \emph{nominative subtyping} of object-oriented languages does not maintain 
the disjointness property, making objects effectively belong to multiple classes. 

Closedness of algebraic data types is particularly useful for reasoning about 
programs by case analysis and allows the compiler to perform an automatic 
\emph{incompleteness} check -- test of whether a given match statement 
covers all possible cases. Similar reasoning about programs involving extensible 
data types is more involved as we are dealing with potentially open set of 
variants. \emph{Completeness} check in such scenario reduces to checking presence 
of a case that handles the static type of the subject. Absence of such a case,
however, does not necessarily imply incompleteness, only potential incompleteness, 
as the answer will depend on the actual set of variants available at run-time.

A related notion of \emph{redundancy} checking arises from the 
tradition of using \emph{first-fit} strategy in pattern matching. It warns the 
user of any \emph{case clause} inside a match statement that will 
never be entered because of a preceding one being more general. Object-oriented 
languages, typically prefer \emph{best-fit} strategy because it is not prone 
to errors where semantics of a statement might change depending on the ordering 
of preceding definitions. 

The patterns used in functions \codeocaml{eval} and \codehaskell{eitherLift} to 
identify and decompose a concrete variant of an algebraic data types are 
generally called \emph{tree patterns} or \emph{constructor patterns}. Their 
analog in object-oriented languages is often referred to as \emph{type pattern} 
since it may involve type testing and type casting. Special cases of tree patterns  
are \emph{list patterns} and \emph{tuple patterns} that make use of special list 
and tuple constructors \codehaskell{:} and \codehaskell{(,,...,)}.

Pattern matching can also be applied to built-in types.
In Haskell, a factorial can be defined by equations:

\begin{lstlisting}[language=Haskell]
factorial 0 = 1
factorial n = n * factorial (n-1)
\end{lstlisting}

\noindent
Here 0 in the left hand side of the first \emph{equation} is an example of a 
\emph{value pattern} that will only match when the actual argument passed is 0. 
The \emph{variable pattern} \codehaskell{n} in the left hand side of the second 
equation will match any value, \emph{binding} variable \codehaskell{n} to it in 
the right hand side of equation. The \emph{wildcard pattern} \codehaskell{_}  
will match any value, neither binding it to a variable nor even obtaining it. 
Value patterns, variable patterns and wildcard patterns are  
generally called \emph{primitive patterns}. Patterns like variable and wildcard 
patterns that never fail to match are called \emph{irrefutable}, in contrast to 
\emph{refutable} patterns like value patterns, which may fail to match.

In Haskell 98\cite{Haskell98Book} factorial could alternatively be defined as:

\begin{lstlisting}[language=Haskell]
factorial 0 = 1
factorial (n+1) = (n+1) * factorial n
\end{lstlisting}

\noindent
The \codehaskell{(n+1)} pattern in the left hand side of equation is an example of 
\emph{n+k pattern}. Accordingly to its informal semantics ``Matching an $n+k$ 
pattern (where $n$ is a variable and $k$ is a positive integer literal) against 
a value $v$ succeeds if $v \ge k$, resulting in the binding of $n$ to $v-k$, and 
fails otherwise''\cite{haskell98}. n+k patterns were introduced into Haskell to 
let users express inductive functions on natural numbers in much the same way as 
functions defined through case analysis on algebraic data types. Besides 
succinct notation, this could facilitate automatic proof of 
termination of such functions by the compiler.
However, numerous debates over semantics and usefulness of n+k patterns
resulted in their removal from the Haskell 
2010 standard\cite{haskell2010}. Generalization of n+k patterns, called 
\emph{application patterns} has been studied by Oosterhof\cite{OosterhofThesis}. 
Application patterns essentially treat n+k patterns as equations, while matching 
against them attempts to solve or validate the equation.

Our library language supports generalized n+k patterns in a different form 
(\textsection\ref{sec:slv}). We do not restrict ourselves with equational view 
of the n+k patterns, but allow the user to specify suitable semantics.
In our library, we can define fast computation of Fibonacci numbers like this:

\begin{lstlisting}[keepspaces]
int fib(int n)
{
    variable<int> m;
    Match(n) {
      When(1)         return 1;     
      When(2)         return 1;
      When(2*m)     return sqr(fib(m+1)) - sqr(fib(m-1));
      When(2*m+1) return sqr(fib(m+1)) + sqr(fib(m));
    } EndMatch
}
\end{lstlisting}

\noindent
A \emph{guard} 
is a predicate attached to a pattern that may make use of the variables bound in 
it. The result of its evaluation will determine whether the case clause and the 
body associated with it will be \emph{accepted} or \emph{rejected}.

In OCaml, we can define rules for rewriting terms in 
our $exp$ language with the help of guards:

\begin{lstlisting}[language=Caml,keepspaces,columns=flexible]
let collect e = match e with
      Plus(Times(e1,e2), Times(e3,e4)) when e1 = e3 -> Times(e1, Plus(e2,e4))
    | Plus(Times(e1,e2), Times(e3,e4)) when e2 = e4 -> Times(Plus(e1,e3), e4)
    | e -> e ;;
\end{lstlisting}

\noindent
Attempting to write the first case clause as \codeocaml{Plus(Times(e,e2), 
Times(e,e4))} would have been relying on \emph{equivalence patterns}. 
Neither OCaml nor Haskell support such patterns, but
Miranda\cite{Miranda85} and Tom\cite{Moreau:2003} do.

The example above illustrates yet another common pattern-matching facility -- 
\emph{nesting of patterns}. Constructor patterns composed 
solely of (distinct) variables are called \emph{simple pattern}s
and others are called \emph{nested pattern}s.
Nested checks are hard to handle using the visitor design pattern, as they are often 
too context-dependent to extract them into a dedicated visitor. 
In such cases, users typically prefer \emph{type tests} and \emph{type 
casts}. Our library handles such cases using nested patterns:
\begin{lstlisting}
const Expr* collect(const Expr* e)
{
  const Expr *e1, *e2, *e3, *e4;
  Match(e) {
    When(match<Plus>(match<Times>(e1,e2),match<Times>(e3 |= e1==e3,e4))) 
        return new Times(e1, new Plus(e2,e4));
    When(match<Plus>(match<Times>(e1,e2),match<Times>(e3,e4 |= e2==e4))) 
        return new Times(new Plus(e1,e3), e4);
    When() 
        return e;
  } EndMatch
}
\end{lstlisting}

\noindent
Decomposing algebraic data types through pattern matching has an important 
drawback that was originally spotted by Wadler\cite{Wadler87}: they expose 
concrete representation of an abstract data type, which conflicts with the 
principle of \emph{data abstraction}. To overcome the problem he proposed the 
notion of \emph{views} that represent conversions between different 
representations that are implicitly applied during pattern matching. As an 
example, imagine polar and cartesian representations of complex numbers. A user 
might choose polar representation as a concrete representation for the abstract 
data type \codeocaml{complex}, treating cartesian representation as view or vice 
versa:\footnote{We use the syntax from Wadler's original paper for this example}

\begin{lstlisting}[language=Haskell,columns=flexible]
complex ::= Pole real real
view complex ::= Cart real real
  in  (Pole r t) = Cart (r * cos t) (r * sin t)
  out (Cart x y) = Pole (sqrt(x^2 + y^2)) (atan2 x y)
\end{lstlisting}

\noindent
The operations then might be implemented in whatever representation is the most 
suitable, while the compiler will implicitly convert representation if needed:

\begin{lstlisting}[language=Haskell,columns=flexible]
  add  (Cart x1 y1) (Cart x2 y2) = Cart (x1 + x2) (y1 + y2)
  mult (Pole r1 t1) (Pole r2 t2) = Pole (r1 * r2) (t1 + t2)
\end{lstlisting}

\noindent
The idea of views were later adopted in various forms in several languages: 
Haskell\cite{views96}, Standard ML\cite{views98}, Scala (in the form of 
\emph{extractors}\cite{EmirThesis}) and F$\sharp$ (under the name of 
\emph{active patterns}\cite{Syme07}). We demonstrate our support of views in 
\textsection\ref{sec:view}.
