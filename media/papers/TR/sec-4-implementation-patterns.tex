\section{Generalized n+k Patterns}
\label{sec:slv}

Intuitively n+k patterns like $f(x,y)=v$ relate a known result of a given 
function application to its arguments. The case where multiple unknown arguments 
are matched against a single result should not be immediately discarded as there 
are known n-ary functions whose inverse is unique. An example of such function 
is Cantor pairing function that defines bijection between 
$\mathbb{N}\times\mathbb{N}$ and $\mathbb{N}$. Even when such mappings are not 
one-to-one, their restriction to a given argument often is. Most generalizations 
of n+k patterns seem to agree on the following rules:

\begin{itemize}
\setlength{\itemsep}{0pt}
\setlength{\parskip}{0pt}
\item Absence of solution that would result in a given value should be indicated 
      through rejection of the pattern.
\item Presence of a unique solution should be indicated with acceptance of the 
      pattern and binding of corresponding variables to the solution.
\end{itemize}

\noindent When multiple solutions are possible, returning a set or an enumerator 
is not usually considered due to differences in types: variable $x$ representing 
a solution to $f(x)=c$ is intuitively expected to have the same type as the 
argument of $f$, so that it can be applied to $x$. Rejecting the pattern might be 
a plausible approach in some applications where multiple solutions are treated as 
ambiguity. It is incapable of distinguishing absence of a solution from ambigous 
solution however, which is often desired.

Binding to an arbitrary solution in case of multiple ones might be sensitive as 
to which solution is chosen: some applications might prefer the 
smallest/largest one, some the smallest positive etc.
We believe that depending on application, different semantic choices can be 
valid, which is why we prefer not to make such choice for the user, but rather 
provide him with means to decide. In fact we go even further and do not require 
the values bound by our generalized n+k pattern to be a solution to the 
corresponding equation. We do this for several reasons:

\begin{itemize}
\item Due to numeric errors, truncation and integer overflow we will rarely 
      obtain the exact algebraic solution.
\item Curve fitting can be seen as pattern matching in some application domains. 
\item Sometimes we might be interested in matching against a projection of a 
      value onto some base and the obtained result may not necessarily yield a 
      solution to matching against the original value.
\end{itemize}

Consider matching an expression $x+1$ with variable $x$ of type \code{char} 
ranging over $[-128,127]$ against a value -128. Should the value 127 be 
considered a solution since 127+1 overflows in char resulting in -128? From the 
mathematical point of view it should not, but a particular application might 
accept such a solution for the sake of performance. Similarly, matching $3*y$ 
against 1.0 with a variable $y$ of type \code{double} will result in a number 
that is slightly different from $\frac{1}{3}$. Should such match be accepted the 
next logical request a user might have is to be able to match an expression of 
the form $n/m$ for integer variables $n$ and $m$ against a value 3.1415926 and 
get the closest fraction to it. Matching against such pattern does not have to 
be imprecise as one can match it against an object of a class representing 
rational numbers.

Taking the argument of precision even further one may want to be able to do 
curve fitting with generalized n+k patterns for the sake of expressive syntax. 
Consider an object that contains sampling of some random variable. A hypotetical 
match statement might be querying:

\begin{lstlisting}[keepspaces,columns=flexible]
match (random_variable) { 
    case Gaussian(@$\mu,\sigma^2$@): ... case Poisson(@$\lambda$@): ... case Bernoulli(@$p$@): ... 
}
\end{lstlisting} 

Fitting error threshold in such scenario can either be global or passed 
as a parameter into expressions we are matching against: e.g. 
\code{case Gaussian(}$0.01,\mu,\sigma^2$\code{):}.
Again the fitting does not have to be imprecise. Consider a library dealing with 
polynomials of arbitrary degree. Given a general polynomial we might want to be 
able to check a few special cases for which analytical solutions to some larger 
question exist:

\begin{lstlisting}[keepspaces,columns=flexible]
match (polynomial) { case a*X^1 + 1: ... case 2*X^2 + b*X^1 + c: ... }
\end{lstlisting} 

X in such scenario is not a variable but a placeholder value of a kind that lets 
us identify the degree, whose coefficient is sought. 

A simpler example of this kind is decomposition of a complex number with Euler's 
notation $a+b*i$ for scalar variables $a$ and $b$. With variables, such a 
generalized n+k pattern is irrefutable for all complex numbers, but when a more 
specific form is queried (e.g. $3+b*i$) a given complex number may fail to match 
such a pattern. While matching, we will project such a complex number along its 
real and imaginary components and will try matching the operands of addition 
using those projections. Solutions obtained along each projection may not 
necessarily combine into the final solution.

What all these examples have in common is not necessarily solving the equation 
that generalized n+k patterns represent, but the fact that we associate certain 
notations with certain mathematical entities they represent. Parameters of 
those expressions are typically associated with the parameterns of the 
underlying mathematical object and we perform decomposition of that object into 
parts. The structure of the expression tree is an analog of a constructor symbol in 
structural decomposition, while its leaves are placeholders for parameters to be 
matched against or inferred from the mathematical object in question.

Algebraic decomposition to mathematical entities is what views are to algebraic 
data types. Consider for example an object representing a 2D line. At different 
parts of the program we might need to decompose that line differently (hypothetical syntax):

\begin{lstlisting}[keepspaces,columns=flexible]
if (line matches m*X + c) ...                       // slope-intercept form
if (line matches a*X + b*Y = c) ...                 // linear equation form
if (line matches (Y-y0)*(x1-x0)=(y1-y0)*(X-x0)) ... // two-points form
\end{lstlisting}

As before, X and Y are not variables, but some syntactic entities that let us 
properly decompose parts. Matching against the slope-intercept notation will not 
be able to decompose a line of the form $y=c$, but otherwise still looks like 
solving an equation (even though quantified over all X). The other two notations 
include equality sign in their expression, which makes our argument that we 
decompose against a known notation (as opposed to solving some equation) 
stronger.

\subsubsection{Solvers}

The above class esssentially defines forward semantics of a family of 
operations. To define backward semantics of it for the use in n+k patterns, the 
user defines \emph{solvers} by overloading a function 

\begin{lstlisting}
template <LazyExpression E, typename S> bool solve(const E&, const S&);
\end{lstlisting}

The first argument of a function takes an expression template representing an 
expression we are matched against, while the second argument represents the 
expected result. The following example defines a generic solver for 
multiplication by a constant:

\begin{lstlisting}
template <LazyExpression E, typename T> requires Field<E::result_type>
bool solve(const expr<multiplication,E,value<T>>& e, const E::result_type& r) {
    return solve(e.m_e1,r/eval(e.m_e2));
}
@\halfline@
template <LazyExpression E, typename T> requires Integral<E::result_type>
bool solve(const expr<multiplication,E,value<T>>& e, const E::result_type& r) {
    T t = eval(e.m_e2);
    return r%t == 0 && solve(e.m_e1,r/t);
}
\end{lstlisting}

\noindent
Note that we overload not only on the structure of the expression, but also on 
the properties of their result type (or any other type involved). In particular 
when the type of the result of the sub-expression models \code{Field} concept, 
we can rely on presence of unique inverse and simply call division without any 
additional checks. A similar overload for integral multiplication additionally 
checks that result is divisible by the constant, before generically forwarding 
the matching to the first argument of multiplication. This last overload 
combined with a similar solver for addition of integral types is everything the 
library needs to properly handle the definition of the \code{fib} function from 
\textsection\ref{sec:syn}.

A solver capable of decomposing a complex value using the Euler's notation is 
very easy to define by fixing the structure of expression:

\begin{lstlisting}
template <LazyExpression E1, LazyExpression E2> requires SameType<E1::result_type,E2::result_type>
bool solve(const expr<addition,
                      expr<multiplication,E1,value<complex<E1::result_type>>>,
                      E2
                     >& e, 
           const complex<E1::result_type>& r);
\end{lstlisting}

\section{Views}
\label{sec:view}

Support of multiple bindings through layouts in our library effectively enables 
a facility similar to Wadler's \emph{views}. Reconsider example from 
\textsection\ref{sec:bg} that discusses cartesian and polar representations of 
complex numbers, demonstrating the notion of view. The same example recoded with 
our SELL looks as following:

\begin{lstlisting}[keepspaces,columns=flexible]
// Introduce layouts
enum { cartesian = default_layout, polar };
@\halfline@
// Define bindings with them
template <typename T> struct bindings<std::complex<T>>
  { CM(0,std::real<T>); CM(1,std::imag<T>); };
template <typename T> struct bindings<std::complex<T>, polar>
  { CM(0,std::abs<T>);  CM(1,std::arg<T>); };
@\halfline@
// Define views
template <typename T> 
  using Cartesian = view<std::complex<T>>;
template <typename T> 
  using Polar     = view<std::complex<T>, polar>;
@\halfline@
  std::complex<double> c;
  double a,b,r,f;
@\halfline@
  if (match<std::complex<double>>(a,b)(c)) // default
  if (match<   Cartesian<double>>(a,b)(c)) // same as above
  if (match<       Polar<double>>(r,f)(c)) // view
\end{lstlisting}

\noindent
The C++ standard effectively enforces the standard library to use cartesian 
representation\cite[\textsection26.4-4]{C++11}. Knowing that, we choose the 
\code{cartesian} layout to be default, with \code{polar} being an alternative 
layout for complex numbers. We then define bindings for each of these layouts as 
well as introduce template aliases (an analog of typedefs for parameterized 
classes) for each of the views. Template class \code{view<T,l>} defined by the 
library provides a way to bind together a target type with one of its layouts 
into a single type. This type can be used everywhere in the library where an 
original target type was expected, while the library will take care of decoding 
the type and layout from the view and passing them along where needed.

The first two match expressions are the same and incur no run-time overhead 
since they use default layout of the underlying type. The third match expression 
will implicitly convert cartesian representation into polar, thus incurring some 
overhead. This overhead would have been present in code that depends on polar 
coordinates anyways, since the user would have had to invoke the corresponding 
functions manually.
