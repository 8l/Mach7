%\section{Pattern Matching for C++} %%%%%%%%%%%%%%%%%%%%%%%%%%%%%%%%%%%%%%%%%%%%%
%\label{sec:pm}

\section{Algebraic Data Types in C++}
\label{sec:adt}

C++ does not have a direct support of algebraic data types, but they can usually 
be emulated in a number of ways. A pattern-matching solution that strives to be 
general will have to account for different encodings and be applicable to all of 
them.

Consider an ML data type of the form:

\begin{lstlisting}[language=ML,keepspaces,columns=flexible,escapechar=@]
datatype DT = @$C_1$@ of {@$L_{11}:T_{11},...,L_{1m}:T_{1m}$@} 
              | ...
              | @$C_k$@ of {@$L_{k1}:T_{k1},...,L_{kn}:T_{kn}$@}
\end{lstlisting}

\noindent There are at least 3 different ways to represent it in C++. Following 
Emir, we will refer to them as \emph{encodings}~\cite{EmirThesis}:

\begin{itemize}
\setlength{\itemsep}{0pt}
\setlength{\parskip}{0pt}
\item Polymorphic Base Class (or \emph{polymorphic encoding} for short)
\item Tagged Class (or \emph{tagged encoding} for short)
\item Discriminated Union (or \emph{union encoding} for short)
\end{itemize}

\noindent
In polymorphic and tagged encoding, base class \code{DT} represents algebraic 
data type, while derived classes represent variants. The only difference between 
the two is that in polymorphic encoding base class has virtual functions, while 
in tagged encoding it has a dedicated member of integral type that uniquely 
identifies the variant -- derived class. 

The first two encodings are inherently \emph{open} because the classes can be 
arbitrarily extended through subclassing. The last encoding is inherently 
\emph{closed} because we cannot add more members to the union without modifying 
its definition.

%In order to be able to provide a common syntax for these representations, we 
%need to understand better similarities and differences between them. Before we 
%look into them let's fix some terminology.

When we deal with pattern matching, the static type of the original expression 
we are matching may not necessarily be the same as the type of expression we 
match it with. We call the original expression a \emph{subject} and its static 
type -- \emph{subject type}. We call the type we are trying to match subject 
against -- a \emph{target type}.

In the simplest case, detecting that the target type is a given type or a type 
derived from it, is everything we want to know. We refer to such a use-case as 
\emph{type testing}. In the next simplest case, besides testing we might want to 
get a pointer or a reference to the target type of subject as casting it to such 
a type may involve a non-trivial computation only a compiler can safely 
generate. We refer to such a use-case as \emph{type identification}. Type 
identification of a given subject against multiple target types is typically 
referred to as \emph{type switching}.

Once we uncovered the target type, we may want to be able to decompose it 
\emph{structurally} (when the target type is a \emph{structured} data type like 
array, tuple or class) or \emph{algebraically} (when the target type is a scalar 
data type like \code{int} or \code{double}). Structural decomposition in our 
library can be performed with the help of \emph{tree patterns}, while algebraic 
decomposition can be done with the help of \emph{generalized n+k patterns}.

\subsection{Polymorphic Base Class}
\label{sec:pbc}

In this encoding user declares a polymorphic base class \code{DT} that will 
be extended by classes representing all the variants. Base class might declare 
several virtual functions that will be overridden by derived classes, for example 
\code{accept} used in a Visitor Design Pattern.

\begin{lstlisting}[keepspaces,columns=flexible]
class DT { virtual @$\sim$@DT{} };
class @$C_1$@ : public DT {@$T_{11} L_{11}; ... T_{1m} L_{1m};$@} 
...
class @$C_k$@ : public DT {@$T_{k1} L_{k1}; ... T_{kn} L_{kn};$@} 
\end{lstlisting}

The uncover the actual variant of such an algebraic data type, the user might 
use \code{dynamic_cast} to query one of the $k$ expected run-time types (an 
approach used by Rose\cite{SQ03}) or she might employ a visitor design pattern 
devised for this algebraic data type (an approach used by Pivot\cite{Pivot09} 
and Phoenix\cite{Phoenix}). The most attractive feature of this approach is that 
it is truly open as we can extend classes arbitrarily at will (leaving the 
orthogonal issues of visitors aside).

\subsection{Tagged Class}
\label{sec:tc}

This encoding is similar to the \emph{Polymorphic Base Class} in that we use 
derived classes to encode the variants. The main difference is that the user 
designates a member in the base class, whose value will uniquely 
determine the most derived class a given object is an instance of. Constructors 
of each variant $C_i$ are responsible for properly initializing the dedicated 
member with a unique value $c_i$ associated with that variant. Clang\cite{Clang} 
among others uses this approach.

\begin{lstlisting}[keepspaces,columns=flexible]
class DT { enum kinds {@$c_1, ..., c_k$@} m_kind; };
class @$C_1$@ : public DT {@$T_{11} L_{11}; ... T_{1m} L_{1m};$@} 
...
class @$C_k$@ : public DT {@$T_{k1} L_{k1}; ... T_{kn} L_{kn};$@} 
\end{lstlisting}

In such scenario the user might use a simple switch statement to uncover the 
type of the variant combined with a \code{static_cast} to properly cast the 
pointer or reference to an object. People might prefer this encoding to the one 
above for performance reasons as it is possible to avoid virtual dispatch with 
it altogether. Note, however, that once we allow for extensions and not limit 
ourselves with encoding algebraic data types only it also has a significant 
drawback in comparison to the previous approach: we can easily check that given 
object belongs to the most derived class, but we cannot say much about whether 
it belongs to one of its base classes. A visitor design pattern can be 
implemented to take care of this problem, but control inversion that comes along 
with it will certainly diminish the convenience of having just a switch 
statement. Besides, forwarding overhead might lose some of the performance 
benefits gained originally by putting a dedicated member into the base class.

\subsection{Discriminated Union}
\label{sec:du}

This encoding is popular in projects that are either implemented in C or 
originated from C before coming to C++. It involves a type that contains a union 
of its possible variants, discriminated with a dedicated value stored as a part 
of the structure. The approach is used by EDG front-end\cite{EDG} and many others.

\begin{lstlisting}[keepspaces,columns=flexible]
struct DT
{
    enum kinds {@$c_1, ..., c_k$@} m_kind;
    union {
        struct @$C_1$@ {@$T_{11} L_{11}; ... T_{1m} L_{1m};$@} @$C_1$@;
        ...
        struct @$C_k$@ {@$T_{k1} L_{k1}; ... T_{kn} L_{kn};$@} @$C_k$@; 
    };
};
\end{lstlisting}

As before, the user can use a switch statement to identify the variant $c_i$ and 
then access its members via $C_i$ union member. This approach is truly closed, as 
we cannot add new variants to the underlying union without modifying class 
definition. 

Note also that in this case both subject type and target types are the same and 
we use an integral constant to distinguish which member(s) of the underlying union 
is active now. In the other two cases the type of a subject is a base class of 
the target type and we use either run-time type information or the integral 
constant associated by the user with the target type to uncover the target type. 

\section{Pattern Matching Syntax}
\label{sec:syn}

Figure~\ref{syntax} presents the syntax enabled by our SELL in an abstract 
syntax form rather than traditional EBNF in order to better describe 
compositions allowed by the library. In particular, the allowed compositions 
depend on the C++ type of the entities being composed, so we need to include it 
in the notation. We do make use of several non-terminals from the C++ grammar in 
order to put the use of our constructs into context.

% TODO:
%()     Function call
%[]     Array subscripting
%*      Indirection (dereference)
%&      Address-of
%sizeof Size-of

\begin{figure}
\begin{center}
\begin{tabular}{rp{0em}cl}
\Rule{match statement}     & $M$       & \is{}  & \code{Match(}$e$\code{)} $\left[C s^*\right]^*$ \code{EndMatch} \\
\Rule{case clause}         & $C$       & \is{}  & \code{Case(}$T\left[,x\right]^*$\code{)} \\
                           &           & \Alt{} & \code{Que(} $T\left[,\omega\right]^*$\code{)} \\
                           &           & \Alt{} & \code{Otherwise(}$\left[,x\right]^*$\code{)} \\
\Rule{target expression}   & $T$       & \is{}  & $\tau$ \Alt{} $l$ \Alt{} $\nu$ \\
\Rule{view}                & $\nu$     & \is{}  & \code{view<}$\tau,l$\code{>} \\
\Rule{match expression}    & $m$       & \is{}  & $\pi(e)$ \\
\Rule{pattern}             & $\pi$     & \is{}  & $\_$ \Alt{} $\eta$ \Alt{} $\varrho$ \Alt{} $\mu$ \Alt{} $\varsigma$ \Alt{} $\chi$ \\
\Rule{extended pattern}    & $\omega$  & \is{}  & $\pi$ \Alt{} $c$ \Alt{} $x$ \\
\Rule{tree pattern}        & $\mu$     & \is{}  & \code{match<}$\nu|\tau\left[,l\right]$\code{>(}$\omega^*$\code{)} \\
\Rule{guard pattern}       & $\varrho$ & \is{}  & $\pi \models \xi$ \\
\Rule{n+k pattern}         & $\eta$    & \is{}  & $\chi$ \Alt{} $\eta \oplus c$ \Alt{} $c \oplus \eta$ \Alt{} $\ominus \eta$ \Alt{} $(\eta)$ \Alt{} $\_$ \\
\Rule{wildcard pattern}    & $\_^{wildcard}$ \\
\Rule{variable pattern}    & $\chi$    & \is{}  & $\kappa$ \Alt{} $\iota$ \\
\Rule{value pattern}       & $\varsigma^{value\langle\tau\rangle}$ \\
\Rule{xt variable}         & $\kappa^{variable\langle\tau\rangle}$ \\
\Rule{xt reference}        & $\iota^{var\_ref\langle\tau\rangle}$  \\
\Rule{xt expression}       & $\xi$     & \is{}  & $\chi$ \Alt{} $\xi \oplus c$ \Alt{} $c \oplus \xi$ \Alt{} $\ominus \xi$ \Alt{} $(\xi)$ \Alt{} $\xi \oplus \xi$ \\
\Rule{layout}              & $l$       & \is{}  & $c^{int}$ \\
\Rule{unary operator}      & $\ominus$ & $\in$  & $\lbrace*,\&,+,-,!,\sim\rbrace$ \\
\Rule{binary operator}     & $\oplus$  & $\in$  & $\lbrace*,/,\%,+,-,\ll,\gg,\&,\wedge,|,$ \\
                           &           &        & $<,\leq,>,\geq,=,\neq,\&\&,||\rbrace$ \\
\Rule{type-id}             & $\tau$    &        & C++\cite[\textsection A.7]{C++0x} \\
\Rule{statement}           & $s$       &        & C++\cite[\textsection A.5]{C++0x} \\
\Rule{expression}          & $e^\tau$  &        & C++\cite[\textsection A.4]{C++0x} \\
\Rule{constant-expression} & $c^\tau$  &        & C++\cite[\textsection A.4]{C++0x} \\
\Rule{identifier}          & $x^\tau$  &        & C++\cite[\textsection A.2]{C++0x} \\
\end{tabular}
\end{center}
\caption{Syntax enabled by out pattern-matching library}
\label{syntax}
\end{figure}

\noindent
{\bf Match statement} is an analog of a switch statement that allows case 
clauses to be used as its case statements. We require it to be terminated with a 
dedicated \code{EndMatch} macro, to properly close the syntactic structure 
introduced with \code{Match} and followed by \code{Case},\code{Que} and 
\code{Otherwise} macros. Match statement will accept subjects of pointer and 
reference types, treating them uniformly in case clauses. This means that user 
does not have to mention \code{*,&} or any of the \code{const,volatile}-qualifiers
when specifying target types. Passing \code{nullptr} as a subject is considered 
\emph{ill-formed} however -- a choice we have made for performance reasons. 
Examples of match statement has already been presented in 
\textsection\ref{sec:intro} and \textsection\ref{sec:xmpl}.

We support three kinds of {\bf case clauses}: \code{Case}-\emph{clause}, 
\code{Que}-\emph{clause} and \code{Otherwise}-\emph{clause} also called 
\emph{default clause}. \code{Case} and \code{Que} clauses are refutable and both 
take a target expression as their first argument. \code{Otherwise} clause is 
irrefutable and can occur at most once among the clauses. Its target type is 
the subject type. \code{Case} and \code{Otherwise} clauses take additionally a 
list of identifiers that will be treated as variable patterns implicitly 
introduced into the clause's scope and bound to corresponding members of their 
target type. \code{Que} clause permits nested patterns as its arguments, but 
naturally requires all the variables used in the patterns to be explicitly 
pre-declared. Even though our default clause is not required to be the last 
clause of the match statement, we strongly encourage the user to place it 
last (hence the choice of name -- otherwise). Placing it at the beginning or in 
the middle of a match statement will only work as expected with \emph{tagged 
class} and \emph{discriminated union} encodings that use 
\emph{the-only-fit-or-default} strategy for choosing cases. The 
\emph{polymorphic base class} encoding uses \emph{first-fit} strategy and thus 
irrefutable default clause will effectively hide all subsequent case 
clauses, making them redundant. As we show in \textsection\ref{} the switch 
between \emph{polymorphic base class} and \emph{tagged class} encodings can 
simply be made with addition or removal a single definition, which may 
inadvertently change semantics of those match statements, where the default 
clause were not placed last.

When default clause takes optional variable patterns, it behaves in 
exactly the same way as \code{Case} clause whose target type is the subject 
type.

{\bf Target expression} used by the case clauses can be either a target type, 
a constant value, representing \emph{layout} (\textsection\ref{sec:bnd}) or a 
\emph{view} type combining the two (\textsection\ref{sec:view}). Constant value 
is only allowed for union encoding of algebraic data types, in which case the 
library assumes the target type to be the subject type.

{\bf Views} in our library are represented by instantiations of a template class 
\code{view<T,l>} that takes a target type and a layout, combining the two into a 
new type. Our library takes care of transparent handling of this new type as the 
original combination of target type and layout. Views are discussed in details 
in~\textsection\ref{sec:view}.

{\bf Match expression} can be seen as an inline version of match statement with 
a single \code{Que}-clause. Once a pattern is created, it can be applied to an 
expression in order to check whether that expression matches the pattern, 
possibly binding some variables in it. The result of application is always of 
type \code{bool} except for the tree pattern, where it is a value convertible to 
\code{bool}. The actual value in this case is going to be a pointer to target 
type \code{T} in case of a successful match and a \code{nullptr} otherwise. 
Match expressions will most commonly be seen to quickly decompose a possibly 
nested expression with a tree pattern as seen in the example in the paragraph 
below. They are the most used expressions under the hood however that let our 
library be composable. 

{\bf Pattern} summarizes \emph{applicative patterns} -- patterns that can be 
used in a match expression described above. For convenience reasons this 
category is extended with $c$ and $x$ to form an {\bf extended pattern} -- a
pattern that can be used as an argument of \emph{tree pattern} and \code{Que} 
clause. Extended pattern lets us use constants as a \emph{value pattern} and 
regular C++ variables as a \emph{variable pattern} inside these constructs. The 
library implicitly recognizes them and transforms into $\varsigma$ and $\iota$ 
respectively. This transformation is further explained in~\textsection\ref{sec:aux} 
with $\stackrel{flt}{\vdash}$ rule set.

{\bf Tree pattern} takes a target type and an optional layout as its template 
arguments, which uniquely determines a concrete decomposition scheme for the 
type. Any nested sub-patterns are taken as run-time arguments. Besides 
applicative patterns, we allow constants and regular C++ variables to be passed 
as arguments to a tree pattern. They are treated as \emph{value patterns} and 
\emph{variable patterns} respectively. Tree patterns can be arbitrarily nested. 
The following example reimplements \code{factorize} from 
\textsection\ref{sec:bg} in C++ enhanced with our SELL:

\begin{lstlisting}
const Expr* factorize(const Expr* e)
{
    const Expr *e1, *e2, *e3, *e4;
    if (match<Plus>(match<Times>(e1,e2),match<Times>(e3,e4))(e))
        if (e1 == e3)
            return new Times(e1, new Plus(e2,e4));
        else
        if (e2 == e4)
            return new Times(new Plus(e1,e3), e4);
    return e;
}
\end{lstlisting}

\noindent
The above example instantiates a nested pattern and then immediately applies it 
to value \code{e} to check whether the value matches the pattern. If it does, 
it binds local variables to sub-expressions, making them available inside if.
Examples like this are known to be a week spot of visitor design pattern and we 
invite you to implement it with it in order to compare both solutions.

{\bf Guard patterns} in our SELL consist of two expressions separated by operator 
\code{|=}\footnote{Operator \code{|=} defining the guard was chosen arbitrarily 
from those that have relatively low precedence in C++. This was done to allow 
most of the other operators be used inside the condition without parenthesis}: 
an expression being matched (left operand) and a condition (right operand). The 
right operand is allowed to make use of the variable bound in the left operand. 
When used on arguments of a tree pattern, the condition is also allowed to make 
use of any variables bound by preceding argument positions. Naturally, a guard 
pattern that follows a tree pattern may use all the variables bound by the tree 
pattern. Consider for example decomposition of a color value, represented as a 
three-byte RGB triplet:

\begin{lstlisting}[keepspaces,columns=flexible]
variable<double> r,g,b;
auto p = match<RGB>( 
             255*r, 
             255*g |= g [<] r, 
             255*b |= b [<] g+r
         ) |= r+g+b <= 0.5;
\end{lstlisting}

\noindent
Note that C++ standard leaves the order of evaluation of functions arguments 
unspecified\cite[\textsection 8.3.6]{C++0x}, while we seem to rely here on 
\emph{left-to-right} order. The reason we can do this lays in the fact that for 
the purpose of pattern matching all the sub-expressions are evaluated lazily and 
the unspecified order of evaluation refers only to the order in which 
corresponding expression templates are created. The actual evaluation of these 
expression templates happens later when the pattern is applied to an expression 
and since at that point we have an entirely built expression at hand, we 
ourselves enforce its evaluation order.

Another important bit about our guards that has to be kept in mind is that 
guards depend on lazy evaluation and thus expression templates. This is why the 
variables mentioned inside a guard pattern must be of type \code{variable<T>} 
and not just \code{T}. Failing to declare them as such will result in eager 
evaluation of guard expression as a normal C++ expression. This will usually go 
unnoticed at compile time, while surprising at run time, especially to novices.

We chose to provide a syntax for guards to be specified on arguments of a tree 
pattern in addition to after the pattern in order to detect mismatches early 
without having to compute and either bind or match subsequent arguments.

{\bf n+k patterns} are essentially a subset of \emph{xt expressions} with at most 
one non-repeated variable in them. This allows for expressions like $2x+1$, but 
not for $x+x+1$, which even though is semantically equivalent to the first one, 
will not be accepted by our library as an \emph{n+k pattern}.

Expressions \code{255*r}, \code{255*g} and \code{255*b} in the example above 
were instances of our \emph{generalized n+k patterns}. Informally they meant the 
following: the value we are matching against is of the form $255*x$, what is the 
value of $x$? Since color components were assumed to be byte values in the range 
$\left[0-255\right]$ the user effectively gets normalized RGB coordinates in 
variables \code{r}, \code{g} and \code{b} ranging over $\left[0.0-1.0\right]$.

n+k pattern are visually appealing in the sense that they let us write code very 
close to mathematical notations often used in literature. Consider the definition 
of fast Fibonacci algorithm taken almost verbatim from the book. Function 
\code{sqr} here returns a square of its argument.

\begin{lstlisting}[keepspaces,columns=flexible]
int fib(int n)
{
    variable<int> m;
    Match(n)
      Que(int,1)     return 1;
      Que(int,2)     return 1;
      Que(int,2*m)   return sqr(fib(m+1)) - sqr(fib(m-1));
      Que(int,2*m+1) return sqr(fib(m+1)) + sqr(fib(m));
    EndMatch
}
\end{lstlisting}

{\bf Wildcard pattern} in our library is represented by a predefined global 
variable \code{_} of a dedicated type \code{wildcard} bearing no state. 
Wildcard pattern is accepted everywhere where a \emph{variable pattern} $\chi$ 
is accepted. The important difference from a use of an unused variable is that 
no code is executed to obtain a value for a given position and copy that value 
into a variable. The position is ignored altogether and the pattern matching 
continues.

There are two kinds of {\bf variable patterns} in our library: \emph{xt variable} 
and \emph{xt reference}. {\bf xt variable} stands for \emph{expression-template 
variable} and refers to variables whose type is \code{variable<T>} for any given 
type \code{T}. {\bf xt reference} is an \emph{expression-template variable 
reference} whose type is \code{var_ref<T>}. The latter is never declared 
explicitly, but is implicitly introduced by the library to wrap regular 
variables in places where our syntax accepts $x$. Both variable kinds are 
terminal symbols in our SELL letting us build expression templates of them. The 
main difference between the two is that {\bf xt variable} $\kappa$ maintains a 
value of type \code{T} as its own state, while {\bf xt reference} $\iota$ only 
keeps a reference to a user-declared variable of type \code{T}. Besides the 
difference in where the state is stored, both kinds of variables have the same 
semantics and we will thus refer to them as $\chi$. We will also use $\chi^\tau$ 
to mean either $\kappa^{variable\langle\tau\rangle}$ or 
$\iota^{var\_ref\langle\tau\rangle}$ since the type of the actual data $\tau$ is 
what we are interested in, while the fact that it was wrapped into 
\code{variable<>} or \code{var_ref<>} can be implicitly inferred from the 
meta-variable $\chi$. We would also like to point out that in most cases a 
variable pattern can be used as regular variable in the same context where a 
variable of type \code{T} can. In few cases where this does not happen 
implicitly, the user might need to put an explicit cast to \code{T&} or 
\code{const T&}.

{\bf Value pattern} is similarly never declared explicitly and is implicitly 
introduced by the library in places where $c$ is accepted.

{\bf xt expression} refers to an \emph{expression-template expression} -- a 
non-terminal symbol in our expression language built by applying a given 
operator to argument expression templates. We use this syntactic category to 
distinguish lazily evaluated expressions introduced by our SELL from eagerly 
evaluated expressions, directly supported by C++.

{\bf Layout} is an enumerator that user can use to define alternative bindings 
for the same class. When layout is not mentioned, the default layout is used, 
which is the only required layout a user has to define if she wishes to make use 
of bindings. We discuss layouts in details in \textsection\ref{sec:bnd}.

{\bf Binary operator} and {\bf unary operator} name a subset of C++ operators we 
make use of and provide support for in our pattern-matching library. 

The remaining syntactic categories refer to non-terminals in the C++ grammar 
bearing the same name. {\bf Identifier} will only refer to variable names in our 
SELL, even though it has a broader meaning in the C++ grammar. {\bf Expression}
subsumes any valid C++ expression. We use expression $e^\tau$ to refer to a C++ 
expression, whose result type is $\tau$. {\bf Constant-expression} is a subset 
of the above restricted to only expression computable at compile time. {\bf 
Statement} refers to any valid statement allowed by the C++ grammar. Our match 
statement $M$ would have been extending this grammar rule with an extra case 
should it have been defined in the grammar directly. {\bf Type-id} represents a 
type expression that designates any valid C++ type. We are using this 
meta-variable in the superscript to other meta-variables in order to indicate a 
C++ type of the entity they represent.

\subsection{Bindings Syntax}
\label{sec:bnd}

Structural decomposition in functional languages is done with the help of 
constructor symbol and a list of patterns in positions that correspond to 
arguments of that constructor. C++ allows for multiple constructors in a class, 
often overloaded for different types but the same arity. Implicit nature of 
variable patterns that matches ``any value'' will thus not help in 
disambiguating such constructors, unless the user explicitly declares the 
variables, thus fixing their types. Besides, C++ does not have means for 
general compile-time reflection, so a library like ours cannot enumerate 
all the constructors present in a class automatically. This is why we decided to 
separate \emph{construction} of objects from their \emph{decomposition} through 
pattern matching with \emph{bindings}.

%Similarly to constructors, a class may have multiple deconstructors. Unlike 
%constructors, deconstructors are named differently.

The following grammar defines a syntax for a sublanguage our user will use to 
specify decomposition of various classes for pattern matching:\footnote{We reuse 
several meta-variables introduced in the previous grammar}

\begin{figure}
\begin{center}
\begin{tabular}{lp{1em}cl}
\Rule{bindings}                &           & \is{}  & $\delta^*$ \\
\Rule{binding definition}      & $\delta$  & \is{}  & \code{template <}$\left[\vec{p}\right]$\code{>} \\
                               &           &        & \code{struct bindings<} $\tau[$\code{<}$\vec{p}$\code{>}$]\left[,l\right]$\code{>} \\
                               &           &        & \code{\{} $\left[ks\right]\left[kv\right]\left[bc\right]\left[cm^*\right]$ \code{\};} \\
\Rule{class member}            & $cm$      & \is{}  & \code{CM(}$c^{size\_t},q$\code{);} \\
\Rule{kind selector}           & $ks$      & \is{}  & \code{KS(}$q$\code{);}    \\
\Rule{kind value}              & $kv$      & \is{}  & \code{KV(}$c$\code{);}    \\
\Rule{base class}              & $bc$      & \is{}  & \code{BC(}$\tau$\code{);} \\
\Rule{template-parameter-list} & $\vec{p}$ &        & C++\cite[\textsection A.12]{C++11} \\
\Rule{qualified-id}            & $q$       &        & C++\cite[\textsection A.4]{C++11} \\
\end{tabular}
\end{center}
\caption{Syntax used to provide bindings to concrete class hierarchy}
\label{fig:bindings}
\end{figure}

\noindent
Any type $\tau$ may have arbitrary amount of \emph{bindings} associated with it 
and distinguished through the \emph{layout} parameter $l$. The \emph{default 
binding} which omits layout papameter is implicitly associated with layout whose
value is equal to predefined constant \code{default_layout = size_t(}$\sim$\code{0)}. 
User-defined layouts should not reuse this dedicated value.

A \emph{Binding definition} consists of either full or partial specialization of a 
template-class:

\begin{lstlisting}
template <typename T, size_t l = default_layout>
    struct bindings;
\end{lstlisting}

\noindent
The body of the class consists a sequence of specifiers, which generate the 
necessary definitions for querying bindings by the library code. Note that 
binding definitions made this way are \emph{non-intrusive} since the original 
class definition is not touched. They also respect \emph{encapsulation} since 
only the public members of the target type will be accessible from within 
\code{bindings} specialization.

A \emph{Class Member} specifier \code{CM(}$c,q$\code{)} that takes (zero-based) binding 
position $c$ and a qualified identifier $q$, specifies a member, whose value will 
be used to bind variable in position $c$ of $\tau$'s decomposition with this 
\emph{binding definition}. Qualified identifier is allowed to be of one of the 
following kinds:

\begin{itemize}
\setlength{\itemsep}{0pt}
\setlength{\parskip}{0pt}
\item Data member of the target type
\item Nullary member-function of the target type
\item Unary external function taking the target type by pointer, reference or value.
\end{itemize}

\noindent
The following example definition provides bindings to the standard library type 
\code{std::complex<T>}:

\begin{lstlisting}[keepspaces,columns=flexible]
template <typename T> struct bindings<std::complex<T>> {
    CM(0, std::complex<T>::real); 
    CM(1, std::complex<T>::imag); 
};
\end{lstlisting}

\noindent
It states that when pattern matching against \code{std::complex<T>} for any 
given type \code{T}, use the result of invoking member-function \code{real()} to 
obtain the value for the first pattern matching position and \code{imag()} for 
the second position. 

In the presence of several overloaded members with the same name but different 
arity, \code{CM} will unambiguously pick one that falls into one of the three 
listed above categories of accepted members. In the example above, nullary 
\code{T std::complex<T>::real() const} is preferred to unary 
\code{void std::complex<T>::real(T)}.

Note that the binding definition above is made once for all instantiations of 
\code{std::complex} and can be fully or partially specialized for cases of 
interest. Non-parameterized classes will fully specialize the \code{bindings} 
trait to define their own bindings.

Using \code{CM} specifier a user defines the semantic functor 
$\Delta_i^{\tau,l},i=1..k$ we introduced in \textsection\ref{sec:semme} as 
following:

\begin{lstlisting}
template <> struct bindings<@$\tau$@> {
    CM(0, @$\tau$@::member_for_position_0); 
    ...
    CM(k, @$\tau$@::member_for_position_k); 
};
\end{lstlisting}

Note that binding definitions made this way are \emph{non-intrusive} since the 
original class definition is not touched. They also respect \emph{encapsulation} 
since only the public members of the target type will be accessible from within 
\code{bindings} specialization.

A \emph{Kind Selector} specifier \code{KS(}$q$\code{)} is used to specify a member 
of the subject type that will uniquely identify the variant for \emph{tagged} 
and \emph{union} encodings. The member $q$ can be of any of the three categories 
listed for \code{CM}, but is required to return an \emph{integral type}.

A \emph{Kind Value} specifier \code{KV(}$c$\code{)} is used by \emph{tagged} and 
\emph{union} encodings to specify a constant $c$ that uniquely identifies the 
variant.

A \emph{Base Class} specifier \code{BC(}$\tau$\code{)} is used by the \emph{tagged}
encoding to specify an immediate base class of the class whose bindings we 
define. A helper \code{BCS(}$\tau*$\code{)} specifier can also be used to 
specify the exact topologically sorted list of base classes 
(\textsection\ref{sec:cotc}).

A \emph{Layout} parameter $l$ can be used to define multiple bindings for the same 
target type. This is particularly essential for \emph{union} encoding where the 
types of the variants are the same as the type of subject and thus layouts 
become the only way to associate variants with position bindings. For this 
reason we require binding definitions for \emph{union} encoding always use the 
same constant $l$ as a kind value specified with \code{KV(l)} and the layout 
parameter $l$!

%The above definition will fall short, however, when we have to define bindings 
%for an algebraic data type encoded as \emph{discriminated union}. In this case 
%we have a single target type and multiple bindings associated with it. To 
%account for this, we recognize that a class (whether discriminated union or not)
%may have alternative bindings associated with it and thus introduce an optional 
%integral parameter called \emph{layout} that can be passed to \code{bindings} to
%distinguish various binding layouts. Classes that are not instances of our 
%discriminated union encoding are free to choose whatever unique constants they 
%feel appropriate to define alternative layouts. We require, however, that classes 
%representing discriminated union encoding use constants that correspond to kinds 
%supported by the union and only access members within layout that are valid for 
%its kind. 

Consider for example the following discriminated union describing various 
shapes:

\begin{lstlisting}[keepspaces,columns=flexible]
struct cloc { double first; double second; };
struct ADTShape
{
    enum shape_kind {circle, square, triangle} kind;
    union {
      struct { cloc center;     double radius; }; // circle
      struct { cloc upper_left; double size; };   // square
      struct { cloc first, second, third; };      // triangle
    };
};

template <> struct bindings<ADTShape> { 
    KS(ADTShape::kind);     // Kind Selector
};
template <> struct bindings<ADTShape, ADTShape::circle> {
    KV(ADTShape::circle);   // Kind Value
    CM(0, ADTShape::center);
    CM(1, ADTShape::radius);
};
\end{lstlisting}

\noindent
\code{KS} specifier within default bindings for \code{ADTShape} tells the library 
that value of a \code{ADTShape::kind} member, extracted from subject at run time, 
should be used to obtain a unique value that identifies the variant. Binding 
definition for \code{circle} variant then uses the same constant 
\code{ADTShape::circle} as the value of the layout parameter of the 
\code{bindings<T,l>} trait and \code{KV(l)} specifier to indicate its \emph{kind 
value}.

Should the shapes have been encoded with a \emph{Tag Class}, the bindings for 
the base class \code{Shape} would have contained \code{KS(Shape::kind)} 
specifier, while derived classes \code{Circle}, \code{Square} and 
\code{Triangle}, representing corresponding variants, would have had 
\code{KV(Shape::circle)} etc. specifiers in their binding definitions. These 
variant classes could have additionally defined a few alternative layouts for 
themselves, in which case the numbers for the layout parameter could have been 
arbitrarily chosen.

\section{Pattern Matching Semantics}
\label{sec:sem}

We use natural semantics\cite{Kahn87} (big-step operational semantics) to 
describe our pattern-matching semantics. As with syntax, we do not formalize the 
semantics of the entire language, but concentrate only on presenting relevant 
parts of our extension. We assume the entire state of the program is modeled by 
an environment $\Gamma$, which we can query as $\Gamma(x)$ to get a value of a 
variable $x$. In addition to meta-variables we have seen already, metavariables 
$u,v$ and $b^{bool}$ range over values. We make a simplifying assumption that 
all values of user-defined types are represented via variables of reference 
types and there exist a non-throwing operation \DynCast{\tau}{v} that can test 
whether an object of a given type is an instance of another type, returning a 
proper reference to it or a dedicated value \nullptr{} that represents 
\code{nullptr}. Intuitively, the semantics of such references is that of 
pointers in C++, which are implicitly dereferenced. We describe our semantics 
with several rule sets that deal with different parts of our syntax.

\subsection{Semantics of Matching Expressions}
\label{sec:semme}

The rule set in Figure~\ref{exprsem} deals with pattern application $\pi(e)$, 
which essentially performs matching of a pattern $\pi$ against expression $e$. 
The judgements are of the form $\Gamma\vdash \pi(e) \evals v,\Gamma'$ that can 
be interpreted as given an environment $\Gamma$, pattern application $\pi(e)$ 
results in value $v$ and environment $\Gamma'$. When we use $\evalspp$ instead 
of $\evals$ we simply pointing out that corresponding evaluation rule comes from 
the C++ semantics and not from our rules.

\begin{figure}
\begin{mathpar}
\inferrule[Wildcard]
{}
{\Gamma\vdash \_(e) \evals true,\Gamma}

\inferrule[Value]
{\Gamma\vdash e \evalspp v,\Gamma_1 \\ \Gamma_1\stackrel{eval}{\vdash} \varsigma \evals u,\Gamma_2}
{\Gamma\vdash \varsigma^\tau(e^\tau) \evals (u==v),\Gamma_2}

\inferrule[Variable]
{\Gamma\vdash e \evalspp v,\Gamma_1 \\ \Gamma_1 \vdash \DynCast{\tau_1}{v} \evalspp u,\Gamma_2}
{\Gamma\vdash \chi^{\tau_1}(e^{\tau_2}) \evals (u \neq \nullptr{}),\Gamma_2[\chi\leftarrow u]}
\end{mathpar}

\begin{mathpar}
\inferrule[n+k Binary Left]
{\Gamma\vdash e \evalspp v_1,\Gamma_1 \\ \Gamma_1\vdash \Psi_\oplus^\tau[v_1](\bullet,c) \evalspp \langle b_2,v_2\rangle,\Gamma_2 \\ \Gamma_2\vdash \eta(v_2) \evals b_3,\Gamma_3}
{\Gamma\vdash (\eta^\tau \oplus c)(e) \evals (b_2 \wedge b_3),\Gamma_3}
\end{mathpar}

\begin{mathpar}
\inferrule[n+k Binary Right]
{\Gamma\vdash e \evalspp v_1,\Gamma_1 \\ \Gamma_1\vdash \Psi_\oplus^\tau[v_1](c,\bullet) \evalspp \langle b_2,v_2\rangle,\Gamma_2 \\ \Gamma_2\vdash \eta(v_2) \evals b_3,\Gamma_3}
{\Gamma\vdash (c \oplus \eta^\tau)(e) \evals (b_2 \wedge b_3),\Gamma_3}
\end{mathpar}

\begin{mathpar}
\inferrule[n+k Unary]
{\Gamma\vdash e \evalspp v_1,\Gamma_1 \\ \Gamma_1\vdash \Psi_\ominus^\tau[v_1](\bullet)  \evalspp \langle b_2,v_2\rangle,\Gamma_2 \\ \Gamma_2\vdash \eta(v_2) \evals b_3,\Gamma_3}
{\Gamma\vdash (\ominus \eta^\tau)(e) \evals (b_2 \wedge b_3),\Gamma_3}
\end{mathpar}

\begin{mathpar}
\inferrule[Guard]
{\Gamma\vdash e \evalspp v_1,\Gamma_1 \\ \Gamma_1\vdash \pi(v_1) \evals b_2,\Gamma_2 \\ \Gamma_2\stackrel{eval}{\vdash} \xi \evals b_3,\Gamma_3}
{\Gamma\vdash (\pi \models \xi)(e) \evals (b_2 \wedge b_3),\Gamma_3}
\end{mathpar}

\begin{mathpar}
\inferrule[Tree-Nullptr]
{\Gamma \vdash e \evalspp v,\Gamma_0 \\ \Gamma_0 \vdash \DynCast{\tau}{v} \evalspp \nullptr{},\Gamma_1}
{\Gamma\vdash ($match$\langle\tau\left[,l\right]\rangle(\omega_1,...,\omega_k))(e) \evals \nullptr{},\Gamma_1}
\end{mathpar}

\begin{mathpar}
\inferrule[Tree-False]
{\Gamma \vdash e \evalspp v,\Gamma_0 \\ \Gamma_0 \vdash \DynCast{\tau}{v} \evalspp u^{\tau},\Gamma_1 \\\\
 \Gamma_1    \vdash \Delta_1    ^{\tau,l}(u) \evalspp v_1,    \Gamma_1'     \\ \Gamma_1'    \stackrel{flt}{\vdash} \omega_1     \evals \pi_1    \\ \Gamma_1'    \vdash \pi_1(v_1)         \evals true, \Gamma_2     \\\\
 \Gamma_2    \vdash \Delta_2    ^{\tau,l}(u) \evalspp v_2,    \Gamma_2'     \\ \Gamma_2'    \stackrel{flt}{\vdash} \omega_2     \evals \pi_2    \\ \Gamma_2'    \vdash \pi_2(v_2)         \evals true, \Gamma_3     \\\\
 \cdots \\\\
 \Gamma_{i-1}\vdash \Delta_{i-1}^{\tau,l}(u) \evalspp v_{i-1},\Gamma_{i-1}' \\ \Gamma_{i-1}'\stackrel{flt}{\vdash} \omega_{i-1} \evals \pi_{i-1}\\ \Gamma_{i-1}'\vdash \pi_{i-1}(v_{i-1}) \evals true, \Gamma_i     \\\\
 \Gamma_i    \vdash \Delta_i    ^{\tau,l}(u) \evalspp v_i,    \Gamma_i'     \\ \Gamma_i'    \stackrel{flt}{\vdash} \omega_i     \evals \pi_i    \\ \Gamma_i'    \vdash \pi_i(v_i)         \evals false,\Gamma_{i+1} \\\\
}
{\Gamma\vdash ($match$\langle\tau\left[,l\right]\rangle(\omega_1,...,\omega_k))(e) \evals \nullptr{},\Gamma_{i+1}}
\end{mathpar}

\begin{mathpar}
\inferrule[Tree-True]
{\Gamma \vdash e \evalspp v,\Gamma_0 \\ \Gamma_0 \vdash \DynCast{\tau}{v} \evalspp u^{\tau},\Gamma_1 \\\\
 \Gamma_1    \vdash \Delta_1    ^{\tau,l}(u) \evalspp v_1,    \Gamma_1'     \\ \Gamma_1'    \stackrel{flt}{\vdash} \omega_1     \evals \pi_1    \\ \Gamma_1'    \vdash \pi_1(v_1)         \evals true, \Gamma_2     \\\\
 \Gamma_2    \vdash \Delta_2    ^{\tau,l}(u) \evalspp v_2,    \Gamma_2'     \\ \Gamma_2'    \stackrel{flt}{\vdash} \omega_2     \evals \pi_2    \\ \Gamma_2'    \vdash \pi_2(v_2)         \evals true, \Gamma_3     \\\\
 \cdots \\\\
%\Gamma_{k-1}\vdash \Delta_{k-1}^{\tau,l}(u) \evalspp v_{k-1},\Gamma_{k-1}' \\ \Gamma_{k-1}'\stackrel{flt}{\vdash} \omega_{k-1} \evals \pi_{k-1}\\ \Gamma_{k-1}'\vdash \pi_{k-1}(v_{k-1}) \evals true, \Gamma_k     \\\\
 \Gamma_k    \vdash \Delta_k    ^{\tau,l}(u) \evalspp v_k,    \Gamma_k'     \\ \Gamma_k'    \stackrel{flt}{\vdash} \omega_k     \evals \pi_k    \\ \Gamma_k'    \vdash \pi_k(v_k)         \evals true, \Gamma_{k+1} \\\\
}
{\Gamma\vdash ($match$\langle\tau\left[,l\right]\rangle(\omega_1,...,\omega_k))(e) \evals u^{\tau},\Gamma_{i+1}}
\end{mathpar}
\caption{Semantics of match-expressions}
\label{exprsem}
\end{figure}

Matching a wildcard pattern against expression always succeeds without changes 
to the environment (\RefTirName{Wildcard}). Matching a value pattern against 
expression succeeds only if the result of evaluating that expression is the same 
as the constant (\RefTirName{Value}). Matching against variable will always 
succeeds when the type of expression $e$ is the same as variable's value type 
$\tau$. When the types are different, the library will try to use 
\code{dynamic_cast<}$\tau$\code{>(e)} to see whether dynamic type of expression 
can be casted to $\tau$. If it does, matching succeeds, binding variable to the 
result of \code{dynamic_cast}. If it does not, matching fails 
(\RefTirName{Variable}).

Our semantics for generalized n+k patterns draws on the notion of 
\emph{backward collecting semantics} used in \emph{abstract 
interpretation}\cite{CousotCousot92-1}. In general, backward collecting 
semantics $Baexp\Sem{A}(E)R$ of an expression $A$ defines the subset of 
possible environments $E$ such that the expression may evaluate, without 
producing a runtime error, to a value belonging to given set $R$:

\begin{lstlisting}
@$Baexp\Sem{A}(E)R = \{x \in E | A(x) \in R\}$@
\end{lstlisting} 

This can be interpreted as following: given a set $E$ from where the arguments 
of an expression $A$ draw their values, as well as a set $R$ of acceptable (not 
all possible) results, find the largest subset $X \subseteq E$ of arguments that 
will only render results of evaluating $A$ on them in $R$.

Intuitively n+k patterns like $f(x,y)=v$ relate a known result of a given 
function application to its arguments (hence analogy with backward collecting 
semantics). The case where multiple unknown arguments are matched against a 
single result should not be immediately discarded as there are known n-ary 
functions whose inverse is unique. An example of such function is Cantor pairing 
function that defines bijection between $\mathbb{N}\times\mathbb{N}$ and $\mathbb{N}$.
Even when such mappings are not one-to-one, their restriction to a given 
argument often is. Most generalizations of n+k patterns seem to agree on the 
following rules:

\begin{itemize}
\item Absence of solution that would result in a given value should be indicated 
      through rejection of the pattern.
\item Presence of a unique solution should be indicated with acceptance of the 
      pattern and binding of corresponding variables to the solution.
\end{itemize}

\noindent As to the case when multiple solutions are possible, several 
alternatives can be viable:

\begin{itemize}
\item Reject the pattern.
\item Accept, binding to either an arbitrary or some normalized solution.
\item When set of solutions is guaranteed to be finite -- accept, binding 
      solutions to a set variable.
\item When set of solutions is guaranteed to be enumerable -- accept, binding 
      solutions to a generator capable of enumerating them all.
\end{itemize}

\noindent
We believe that depending on application, any of these semantic choices can be 
valid, which is why we prefer not to make such choice for the user, but rather 
provide him with means to decide himself. This is why our semantics for matching 
against generalized n+k pattern depends on a family of user-defined functions 

\begin{lstlisting}
@$\Psi_f^\tau:\tau_r\times\tau_1\times...\times1\times...\times\tau_k\rightarrow bool\times\tau$@
\end{lstlisting} 

\noindent
such that $\Psi_f^\tau[r](c_1,...,\bullet,...,c_k)$ for a given function 

\begin{lstlisting}
@$f:\tau_1,...,\tau,...,\tau_k\rightarrow\tau_r$@
\end{lstlisting} 

\noindent should return a pair composed of a boolean indicating acceptance of a 
pattern $f(c_1,...,x^\tau,...,c_k)=r$ and a solution with respect to $x$ 
when the match was reported successful. Symbol $\bullet$ indicates a position in 
the argument list for which a solution is sought, the rest of the arguments are 
known values. We describe how user can supply the function $\Psi_f^\tau$ for his 
own operation $f$ in \textsection\ref{sec:slv}.

The only difference between the three rules defining semantics of n+k patterns 
is the arity of the root operator and the position in the argument list in which 
the only non-value pattern was spotted (\RefTirName{n+k Binary Left}, 
\RefTirName{n+k Binary Right} and \RefTirName{n+k Unary}). We use user-defined 
$\Psi_f^\tau$ to solve for a given argument of a given operator and then recurse 
to match corresponding sub-expression. Note that when user did not provide 
$\Psi_f^\tau$ for the argument in which solution is sought, the rule is 
rejected.

Evaluation of a guard pattern first tries to match the left hand side of a guard 
expression, usually binding a variable in it, and then if the match was 
successful, lazily evaluates its right hand side to make sure the new value of 
the bound variable is used. The result of evaluating the right hand side 
converted to \code{bool} is reported as the result of matching the entire 
pattern (\RefTirName{Guard}).

Matching of a tree patterns begins with evaluating subject expression and 
casting it dynamically to target type $\tau$ when subject type is not $\tau$. 
When \code{dynamic_cast} fails returning \code{nullptr}, the pattern is rejected 
(\RefTirName{Tree-Nullptr}). Once a value of a target type has been uncovered, 
we proceed with matching arguments left-to-right. For each argument we first 
translate \emph{extended pattern} $\omega$ accepted by tree pattern into an 
\emph{application pattern} $\pi$ to get rid of the syntactic convenience we 
allow on arguments of tree patterns. Using the target type and optional layout 
that was provided by the user, we obtain a value that should be bound in the 
$i^{th}$ position of decomposition. Again we rely on family of user provided 
functions $\Delta_i^{\tau,l}(u)$ that take an object instance and returns a 
value bound in the $i^{th}$ position of its $l^{th}$ layout. Specification of 
function $\Delta$ by the user is discussed in~\textsection\ref{sec:bnd}. Here we 
would like to point out however that the number of argument patterns passed over 
to the tree pattern could be smaller than the number of binding positions defined 
by specific layout. Remaining argument positions are implicitly assumed to be 
filled with the wildcard pattern.

When we have the value for the $i^{th}$ position of object's decomposition, we 
match it against the pattern specified in the $i^{th}$ position and if the value 
is accepted we move on to matching the next argument (\RefTirName{Tree-False}). 
Only in case when all the argument patterns have been successfully matched the 
matching succeeds by returning a pointer to the target type, which in C++ can be 
used everywhere a boolean expression is expected. Returning pointer instead of 
just boolean value gives us functionality similar to that of \emph{as 
patterns} while maintaining the composibility with the 
rest of the library (\RefTirName{Tree-True}).

\subsection{Semantics of Match Statement}
\label{sec:semms}

Our second rule set deals with semantics of a \emph{match statement}. The 
judgements are of the form $\Gamma\vdash s \evals u,\Gamma'$ on statements, 
including match statement, and are slightly extended for case clauses 
$\Gamma\vdash_v C \evals u,\Gamma'$ with value $v$ of a subject that is passed 
along from the match statement onto the clauses. We also use a small helper 
function $TL(t,\tau_s)$ defined on target expression $t \in T$ and the subject's 
type $\tau_s$:
\begin{eqnarray*}
    TL(\tau,\tau_s)                     &=& \langle \tau,  default\_layout \rangle \\
	TL(l,\tau_s)                        &=& \langle \tau_s, l \rangle \\
	TL(view\langle\tau,l\rangle,\tau_s) &=& \langle \tau,   l \rangle
\end{eqnarray*}
\noindent
The function essentially disambiguates one of the three kinds of target 
expressions and returns a combination of a target type and layout used in each 
case.

%\begin{figure}
\begin{mathpar}
\inferrule[Match-True]
{\Gamma \vdash e \evalspp v,\Gamma_1 \\ v \neq \nullptr \\
 \Gamma_1    \vdash_v C_1    \evals false,\Gamma_2 \\
 \Gamma_2    \vdash_v C_2    \evals false,\Gamma_3 \\
 \cdots \\
 \Gamma_{i-1}\vdash_v C_{i-1}\evals false,\Gamma_i \\
 \Gamma_i    \vdash_v C_i    \evals true, \Gamma_{i+1} \\
 \Gamma_{i+1}\vdash \vec{s}_i \evalspp u,\Gamma'
}
{\Gamma\vdash Match(e) \left[C_i \vec{s}_i\right]^*_{i=1..n} EndMatch \evals u,\Gamma'$\textbackslash$\{x | x \not\in \Gamma_i\}}

\inferrule[Match-False]
{\Gamma \vdash e \evalspp v,\Gamma_1 \\ v \neq \nullptr \\
 \Gamma_1    \vdash_v C_1    \evals false,\Gamma_2 \\
 \Gamma_2    \vdash_v C_2    \evals false,\Gamma_3 \\
 \cdots \\
 \Gamma_{n-1}\vdash_v C_{n-1}\evals false,\Gamma_n \\
 \Gamma_n    \vdash_v C_n    \evals false,\Gamma_{n+1}
}
{\Gamma\vdash Match(e) \left[C_i \vec{s}_i\right]^*_{i=1..n} EndMatch \evals false,\Gamma_{n+1}}

\inferrule[Que]
{TL(t,\sigma)=\langle \tau,l \rangle \\
 \Gamma \vdash $match$\langle\tau,l\rangle(\vec{\omega})(v) \evals u,\Gamma' \\
 \Gamma'' = (u \neq \nullptr\ ?\ \Gamma'[$matched$^\tau\rightarrow u] : \Gamma')}
{\Gamma \vdash_{v^\sigma} Que(t,\vec{\omega})    \evals u,\Gamma''}

\inferrule[Case]
{\Delta_i^t : \tau \rightarrow \tau_i, i=1..k \\
 \Gamma[x_i^{\tau_i}\rightarrow\tau_i()]_{i=1..k} \vdash_v Que(t,x_1,...,x_k)(v) \evals u,\Gamma' \\
 \Gamma'' = (u \neq \nullptr\ ?\ \Gamma' : \Gamma'$\textbackslash$\{x_i | i=1..k\})}
{\Gamma \vdash_v Case(t,x_1,...,x_k) \evals u,\Gamma''}

\inferrule[Otherwise]
{\Gamma \vdash Case(\tau,\vec{x})(v) \evals true,\Gamma'}
{\Gamma \vdash_{v^\tau} Otherwise(\vec{x}) \evals true,\Gamma'}
\end{mathpar}
%\caption{Semantics of match-statement}
%\label{stmtsem}
%\end{figure}

\noindent
Evaluation of a match statement begins with evaluation of subject expression, 
which is not allowed to result in \code{nullptr}. This value is passed along to 
each of the clauses. The clauses are evaluated in their lexical order until the
first one that is not rejected. Statements associated with it are evaluated to 
form the outcome of the match statement. The resulting environment makes sure 
that local variables introduced by case clauses are not available after the 
match statement (\RefTirName{Match-True}). When none of the clauses were 
accepted, which is only possible when default clause was not specified, the 
resulting environment might still be different from the initial environment 
because of variables bound in partial matches during evaluation of clauses 
(\RefTirName{Match-False}).

Evaluation of a \code{Que}-clause is equivalent to evaluation of a corresponding 
match-expression on a tree pattern. Successful match will introduce a variable 
\code{matched} of type $\tau\&$ that is bound to subject properly casted to the 
target type $\tau$ into the local scope of the clause.

Evaluation of \code{Case}-clauses amounts of evaluation of \code{Que}-clauses in 
the environment extended with variables passed as arguments to the clause. The 
variables introduced by the \code{Case}-clause have the static type of values 
bound in corresponding positions, which ensures that variable patterns will be 
irrefutable (\RefTirName{Case}). In practice, the variables are of reference 
type so that no unnecessary copying is happening.

Evaluation of default clause cannot fail because there is no \code{dynamic_cast} 
involved neither for the subject nor for implicit local variables: for the 
former the target type is by definition the subject type, while for the latter 
the type is chosen to be the type of expected values (\RefTirName{Otherwise}).

\subsection{Auxiliary Rules}
\label{sec:aux}

The next rule set deals with evaluation of expression templates referred to from 
the previous rule sets via $\stackrel{eval}{\vdash}$. The judgements are of the 
form $\Gamma\stackrel{eval}{\vdash} \xi \evals v,\Gamma'$ that can be 
interpreted as given an environment $\Gamma$, evaluation of an expression 
template $\xi$ results in value $v$ and environment $\Gamma'$. We refer here to 
an unspecified semantic function $\Sem{o}$ that represents C++ semantics of 
operation $o$ as specified by the C++ standard.

\begin{mathpar}
\inferrule[Var]
{}
{\Gamma\stackrel{eval}{\vdash} \chi \evals \Gamma(\chi),\Gamma}

\inferrule[Unary]
{\Gamma\stackrel{eval}{\vdash} \xi \evals v,\Gamma_1}
{\Gamma\stackrel{eval}{\vdash} \ominus \xi \evals \Sem{\ominus} v,\Gamma_1}

\inferrule[Binary]
{\Gamma\stackrel{eval}{\vdash} \xi_1 \evals v_1,\Gamma_1 \\ \Gamma_1\stackrel{eval}{\vdash} \xi_2 \evals v_2,\Gamma_2}
{\Gamma\stackrel{eval}{\vdash} \xi_1 \oplus \xi_2 \evals v_1\Sem{\oplus}v_2,\Gamma_2}

\inferrule[Binary-Left]
{\Gamma\stackrel{eval}{\vdash} \xi \evals v,\Gamma_1}
{\Gamma\stackrel{eval}{\vdash} \xi \oplus c \evals v\Sem{\oplus}c,\Gamma_1}

\inferrule[Binary-Right]
{\Gamma\stackrel{eval}{\vdash} \xi \evals v,\Gamma_1}
{\Gamma\stackrel{eval}{\vdash} c \oplus \xi \evals c\Sem{\oplus}v,\Gamma_1}
\end{mathpar}

\noindent
The rules are quite simple so we do not elaborate them in details. The reason we 
have two separate rules for the case when one of the arguments is constant 
expression stems from the idiomatic use of expression templates enabling direct 
use of constants in operations that already involve expression template 
arguments.

The next set of rules describes transformation of extended patterns into 
applicative patterns to get rid of syntactic sugar enabled by extended patterns. 

\begin{mathpar}
\inferrule[Filter-Pattern]
{}
{\Gamma\stackrel{flt}{\vdash} \pi \evals \pi}

\inferrule[Filter-Variable]
{}
{\Gamma\stackrel{flt}{\vdash} x \evals \iota(x)}

\inferrule[Filter-Constant]
{}
{\Gamma\stackrel{flt}{\vdash} c \evals \varsigma(c)}
\end{mathpar}
