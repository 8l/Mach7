\section{Memoized Dynamic Cast}
\label{sec:memcast}

We saw in Corollary~\ref{crl:vtbl} that the results of \code{dynamic_cast} can 
be reapplied to a different instance from within the same subobject. This leads 
to simple idea of memoizing the results of \code{dynamic_cast} and then using 
them on subsequent casts. In what follows we will only be dealing with the 
pointer version of the operator since the version on references that has a 
slight semantic difference can be easily implemented in terms of the pointer one.

The \code{dynamic_cast} operator in \Cpp{} involves two arguments: a value argument 
representing an object of a known static type as well as a type argument 
denoting the runtime type we are querying. Its behavior is twofold: on one hand 
it should be able to determine when the object's dynamic type is not a 
subtype of the queried type (or when the cast is ambiguous), while on the other 
it should be able to produce an offset by which to adjust the value argument when it is.

We mimic the syntax of \code{dynamic_cast} by defining:

\begin{lstlisting}
template <typename T, typename S>
inline T memoized_cast(S*);
\end{lstlisting}

\noindent
which lets the user replace all the uses of \code{dynamic_cast} in the program 
with \code{memoized_cast} with a simple:

\begin{lstlisting}
#define dynamic_cast memoized_cast
\end{lstlisting}

\noindent
It is important to stress that the offset is not a function of the source and target 
types of the \code{dynamic_cast} operator, which is why we cannot simply memoize the 
outcome inside the individual instantiations of \code{memoized_cast}.
The use of repeated multiple inheritance will result in classes having several 
different offsets associated with the same pair of source and target types 
depending on which subobject the cast is performed from. According to 
corollary~\ref{crl:vtbl}, however, it is a function of target type and the value 
of the vtbl-pointer stored in the object, because the vtbl-pointer uniquely 
determines the subobject within the dynamic type. Our memoization of the 
results of \code{dynamic_cast} should thus be specific to a vtbl-pointer and the 
target type. 

The easiest way to achieve this would be to use a dedicated global
\code{vtblmap<std::ptrdiff_t>} (\textsection\ref{sec:sovtp}) per each 
instantiation of the \code{memoized_cast}. This, however, will create an 
unnecessarily large amount of vtblmap structures, many of which will be  
duplicating information and repeating the work already done. This will happen 
because instantiations of \code{memoized_cast} with same target but different 
source types can share their vtblmap structures since vtbl pointers of different 
source types are necessarily different according to Theorem~\ref{thm:vtbl}. 

Even though the above solution can be easily improved to allocates a single 
vtblmap per target type, an average application might have a lot of different 
target types. This is especially true for applications that will use our Match 
statement since we use \code{dynamic_cast} under the hood in each case clause. 
Indeed our C++ pretty printer was creating 160 vtblmaps of relatively small size 
each, which was increasing the executable size quite significantly because of 
numerous instantiations as well as noticably slowed down the compilation time.

To overcome the problem we turn each target type into a runtime instantiation 
index of the type and allocate a single \code{vtblmap<std::vector<std::ptrdiff_t>>} 
that associates vtbl pointers with a vector of offsets indexed by target type. 
The slight performance overhead that is brought by this improvement is specific 
to our library solution and would not be present in a compiler implementaion. 
Instead we get a much smaller memory footrpint, which can be made even smaller 
once we recognize the fact that global type indexing may effectively enumerate 
target classes that will never appear in the same Match statement. This will 
result in entries in the vector of offsets that are never used.

Our actual solution uses separate indexing of target types for each source type 
they are used with, and also allocates a different 
\code{vtblmap<std::vector<std::ptrdiff_t>>} for each source type. This lets us 
minimize unused entries within offset vectors by making sure only the plausible 
target types for a given source type are indexed. This solution should be 
suitable for most applications since we expect to have a fairly small 
number of source types for the \code{dynamic_cast} operator and a much larger number 
of target types. For the unlikely case of a small number of target types and large 
number of source types we allow the user to revert to the default behavior with a 
library configuration switch that allocates a single \code{vtblmap} per target type as 
we have already discussed above.

The use of \code{memoized_cast} to implement the \code{Match}-statement potentially reuses the 
results of \code{dynamic_cast} computations across multiple independent match 
statements. This allows leveraging the cost of the expensive first call with a 
given vtbl-pointer even further across all the match statements inside the 
program. The above define, with which a user can easily turn all dynamic casts 
into memoized casts, can be used to speed-up existing code that uses dynamic 
casting without any refactoring overhead.
