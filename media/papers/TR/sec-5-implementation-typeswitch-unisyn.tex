\section{Unified Syntax}
\label{sec:unisyn}

The discussion in this subsection will be irrelevant for a compiler 
implementation, nevertheless we include it because some of the challenges we 
came accross as well as techniques we used to overcome them might show up in 
other active libraries. The problem is that working in a library setting, the 
toolbox of properties we can automatically infer about user's class hierarchy, 
match statement, clauses in it, etc. is much more limited than the set of 
properties a compiler can infer. On one side such additional information may let 
us generate a better code, but on the other side we understand that it is 
important not to overburden the user's syntax with every bit of information she 
can possibly provide us with to generate a better code. Some examples of 
information we can use to generate a better code even in the library setting 
include:

\begin{itemize}
\setlength{\itemsep}{0pt}
\setlength{\parskip}{0pt}
\item Encoding we are dealing with (\textsection\ref{sec:adt})
\item Shape of the class hierarchy: flat/deep, single/multiple inheritance etc.
\item The amount of clauses in the match statement
\item Presense of Otherwise clause in the match statement
\item Presence of extensions in dynamically linked libraries
\end{itemize}

We try to infer the information when we can, but otherwise resort to a usually 
slower default that will work in all or most of the cases. The major source of 
inefficiency comes from the fact that macro resolution happens before any 
meta-programming techniques can be employed and thus the macros have to generate 
a syntactic structure that can essentially handle all the cases as opposed to 
the exact case. Each of the macros involved in rendering the syntactic structure 
of a match statement (e.g. \code{Match}, \code{Case}, \code{Otherwise}) have a 
version identified with a suffix that is specific to a combination of encoding 
and shape of the class hierarchy. By default the macros are resolved to a 
unified version that infers encoding with a template meta-program, but this 
resolution can be overriden with a configuration flag for a more specific 
version when all the match statements in user's program satisfy the requirements 
of that version. The user can also pin-point specific match statement with the 
most applicable version, but we discourage such use as performance differences 
are not big enough to justify the exposure of details.

To better understand what is going on, consider the following examples. Case 
labels for polymorphic base class encoding can be arbitrary, but preferably 
sequential numbers, while the case labels for tagged class and discriminated 
union encodings are the actual kind values associated with concrete variants.
Discriminated union and tagged class encodings can use both types (views in case
of unions) and kind values to identify the target variant, while polymorphic 
base class encoding can only use types for that. The latter encoding requires 
allocation of a static vtblmap in each match statement, not needed by any other 
encoding, while tagged class encoding on non-flat hierarchy requires the use of 
default label of the generated switch statement as well as a dedicated case 
label distinct from all kind values (\textsection\ref{sec:cotc}). 
When merging these and other requirements into a syntactic structure of a 
unified version capable of handling any encoding we essentially always have to 
reserve the use of default label (and thus not use it to generate 
\code{Otherwise}-clause), allocate an extra dedicated case label, introduce  
a loop over base classes used by tagged class encoding etc. This is a clear 
overhead for handling of a discriminated union encoding whose syntactic 
structure only involves a simple switch over kind values and default label to 
implement \code{Otherwise}. To minimize the effects of this overhead we rely on 
compiler's optimizer to inline calls specific to each encoding and either remove 
branching on conditions that will always be true after inlining or elminate dead 
code on conditions that will always be false after inlining. Luckily for us 
today's compilers do a great job in doing just that, rendering our unified 
version only slightly less efficient than the specialized ones. These 
differences can be best seen in Figure\ref{relperf} under corresponding entries 
of \emph{Unified} and \emph{Specialized} columns.
