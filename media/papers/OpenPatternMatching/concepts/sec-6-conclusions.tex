\section{Conclusions and Future Work} %%%%%%%%%%%%%%%%%%%%%%%%%%%%%%%%%%%%%%%%%%
\label{sec:cc}

We present a pattern-matching library for \Cpp{} that provides fairly standard
pattern-matching facilities. Our solution is non-intrusive and can be 
retroactively applied. %to any polymorphic or tagged  
%class hierarchy. It also provides a uniform notation to these different 
%encodings of algebraic and extensible hierarchical data types in \Cpp{}.
The library provides efficient and expressive matching on multiple subjects and 
compares to multiple dispatch alternatives in terms of both time and space.

We generalize n+k patterns to arbitrary expressions by letting the user define 
the exact semantics of such patterns. Our approach is more general than traditional approaches 
as it does not require an
equational view of such patterns. It also avoids hardcoding the 
exact semantics of n+k patterns into the language. 

We used the library to rewrite existing code that relied heavily on the 
visitor design pattern.
Our pattern matching code was much shorter (both source and object code), 
simpler, easier to maintain, comprehend, and faster. 
This confirmed our view of the visitor pattern as a clever workaround,
rather than a good solution to a fundamental problem.
The library approach was essential 
for experimentation in the context of real programs and for delivering 
performance comparable with or superior to conventional techniques in the 
context of industrial compilers.

The work presented here continues our research on pattern matching for 
\Cpp{}~\cite{TS12}. We plan to further experiment with other kinds of patterns, 
including those defined by the user, look at the interaction of patterns with 
other facilities in the language and the standard library, and make views less 
ad hoc. For example, standard containers in \Cpp{} do not have the implicit 
recursive structure present in data types of functional languages, and viewing 
them as such with views would incur significant overheads. We will experiment 
with very general patterns as first-class citizens.

Our generalization of n+k patterns depends on the properties of types involved 
in the expression. This should let us experiment not only with generic 
functions, but also with their generic inversions in the form of solvers. As 
more \Cpp{}11 features become available in compilers it will also be interesting to 
look at how use of these features affects the ease of use, performance, 
readability, writability and debugging of the library and the user code that 
uses it.

%In the nearest future, we would like to make our library to be safe and efficient 
%in a multi-threaded environment. 

%From Morten Rhiger:
%New languages are often constructed by piling new features on top of an existing language's de?nition and by integrating these features in the existing language's implementation. However, it is a sign of expressiveness if new features can be implemented within an
%existing language without changing its de?nition.
%Short of macros, functional languages such as Haskell and Standard ML require new
%features to be expressed in terms of typed higher-order functions. We have demonstrated
%how to extend - or, in Guy Steele's terminology, to "grow" (Steele Jr., 1999) - Haskell
%with our own statically typed implementation of pattern matching and we have shown how
%to extend this framework with patterns not currently supported by Haskell.
